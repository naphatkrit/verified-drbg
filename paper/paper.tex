\documentclass[pageno]{jpaper}

\newcommand{\IWreport}{2015}

%correct bad hyphenation here
\hyphenation{Comp-Cert}
\hyphenation{Comp-Cert-TSO}
\usepackage[normalem]{ulem}
\usepackage{listings}
\usepackage{lstlangcoq}

\lstset{language=Coq,basicstyle=\sffamily,mathescape=true,columns=fullflexible}

\newcommand{\stdtitle}[1]{\textbf{#1}}

\begin{document}

\title{
Verification of a Deterministic Random Bits Generator
}

\author{Naphat Sanguansin\\Adviser: Andrew W. Appel}

\date{}
\maketitle

\thispagestyle{empty}
\doublespacing
\begin{abstract}
In this project, I formally verified the \stdtitle{mbedTLS} implementation of a HMAC-based deterministic random bits generator (DRBG) with respect to the standard \stdtitle{NIST SP 800-90A}. By “formal”, I mean the verification is in the machine-checked environment of Coq using the Verified Software Toolchain (VST) framework. I verified the functions \lstinline{update}, \lstinline{reseed}, and \lstinline{generate}. Due to time constraint, I was not able to verify the \lstinline{instantiate} function.
\end{abstract}

\section{Introduction}
A DRBG is a program that takes a source of randomness, or entropy, and stretches it out into a string of pseudorandom bits. \stdtitle{NIST SP 800-90A} \cite{nist} DRBG provides backtracking resistance and, optionally, prediction resistance. Backtracking resistance means that even if the whole (secret) state of the DRBG is known at time $t$, the state at time $t - 1$ cannot be computed. Note that this implies the state at time $t - i$ cannot be computed for all $i \geq 1$. Prediction resistance means that even if the whole (secret) state of the DRBG is known at time $t$, the state at time $t + 1$ cannot be computed. Again, note that this implies the state at time $t + i$ cannot be computed for all $i \geq 1$. Here, “time” is counted in terms of calls to the \lstinline{generate} function.

\stdtitle{NIST SP 800-90A} specifies different types of DRBG. The version I verified in this project is the one based on the key-hashed message authentication code, or HMAC. HMAC takes a (secret) key and a message and computes a fixed-length message authentication code. In theory, someone with only the message will not be able to compute the message authentication code, and someone with both the message and the message authentication code will not be able to compute the key.

Why did I choose to verify an implementation of DRBG? In short, DRBG is a widely used crypto primitive. Correctly implemented crypto primitives help application programmers improve the security of their code without them having to fully understand the math behind the crypto algorithms.

\subsection{Machine-Checked Proofs}
I did the verification in Coq, an open-sourced proof assistant. The C code is compiled into its abstract syntax tree (AST) in Coq using the CompCert certified (machine-checked) compiler. I wrote functional specifications (functional programs) in Coq that closely match \stdtitle{NIST SP 800-90A}. Using the functional programs, I created API specifications using the VST framework. API specifications reason about how the AST deals with memory and local variables. I then prove the AST correct with respect to the API specifications.

Of what the tasks I mentioned, the ones that must be trusted are the translation of \stdtitle{NIST SP 800-90A} into functional specifications and the creation of API specifications from those functional programs. Everything else is machine-checked. Can the functional and API specifications be trusted? For the latter, I hope to make a convincing claim by the end of this paper that as long as the functional specifications are correct, the corresponding API specifications will be, too. As for the functional specifications, first of all, they are small, easy-to-reason-about functional programs. I also ran the programs on some test inputs from NIST and verified that the output matches. While this is already a huge step forward from a C program, as a functional program in Coq is less error-prone, it is still unsatisfying that at the functional specification level, I had to resort to unit tests. Thankfully, there is a parallel project by Katherine Ye ‘16 that will prove cryptographic properties of the functional specifications. The fact that this project can be done in parallel to (and mostly independent from) the verification of the C implementation shows the power and flexibility of VST.

\subsection{Related Works}
Similar verifications have been done for SHA-256 and HMAC, see \cite{sha} and \cite{hmac} respectively. As mentioned previously, Katherine Ye ‘16 is working on proving cryptographic properties of the functional specifications.

As far as I could tell, this is the first end-to-end verification of an implementation of HMAC DRBG against a functional specification. By “end-to-end” I mean that the verification builds upon an existing verification of HMAC \cite{hmac} (which is, in turn, built upon an existing verification of SHA-256 \cite{sha}).

There have been some other verification efforts to verify a DRBG, albeit without using a functional specification. As an example, \cite{prng} verified Android’s pseudorandom number generator. This verification is done on a Java implementation, and is a verification on information flow, not on a functional specification. That is, the project proves that all the bytes of entropy are used in constructing the secret internal state. Proving an implementation against a functional specification is arguably better, as it allows the functional specification to serve as a mathematical representation of the more complex C (or Java) implementation. This functional specification can then be shown to have desired cryptographic properties, including the fact that all bytes of entropy are being properly consumed.


\subsection{Why mbedTLS?}

Previous efforts in the VST project to verify SHA-256 \cite{sha} and HMAC \cite{hmac} have done so using \stdtitle{OpenSSL 0.9.1c}. A natural choice for the DRBG verification would be to also use the same \stdtitle{OpenSSL} library. Unfortunately, the DRBG implementation in \stdtitle{OpenSSL} makes heavy use of function pointers to encapsulate the differences between each type of DRBG. While Verifiable C is capable of reasoning about function pointers, the proof automation system is not yet robust enough when dealing with function pointers and so such proofs will be more complex. \stdtitle{mbedTLS}, on the other hand, does not make use of function pointers except to deal with entropy. Therefore, I chose to use the implementation of DRBG found in \stdtitle{mbedTLS 2.1.1}.

\subsection{Modifications to mbedTLS}
I had to make a few modifications to the \stdtitile{mbedTLS} DRBG source file. Most of these transformations are obviously equivalent to the original version. For example, Verifiable C does not allow memory dereferences in function parameters or if/while/for-loop conditions.

Therefore, code such as the following had to be changed:

\begin{lstlisting}[language=C]
    md_len = mbedtls_md_get_size( ctx->md_ctx.md_info );
\end{lstlisting}

to this:

\begin{lstlisting}[language=C]
    info = ctx->md_ctx.md_info;
    md_len = mbedtls_md_get_size( info );
\end{lstlisting}

The other modification I had to make was to eliminate the use of function pointers when dealing with entropy.

\section{A Functional Representation of Entropy} \label{entropy}
Before even understanding what a DRBG does, I need to explain how entropy is represented.

A DRBG needs some entropy input to act as a source of randomness. According to \stdtitle{NIST SP 800-90A}, the source of entropy input can be either an approved entropy source, a nondeterministic random bits generator (NRBG), or another DRBG. The entropy API used in NIST is the following:

\begin{lstlisting}
    (status, entropy_input) = Get_entropy_input(min_entropy, min_length, max_length, prediction_resistance_request)
\end{lstlisting}

\lstinline{min_entropy} refers to how many bits of entropy to use. \lstinline{min_length} and \lstinline{max_length} bounds the length of the output string of bits. If \lstinline{prediction_resistance_request} is set, then fresh bits of entropy are used.

For this project, I wanted a simple entropy representation that still satisfies \stdtitle{NIST SP 800-90A}. A flawed idea is to represent entropy as an infinite stream of \lstinline{bool}, each \lstinline{bool} representing a bit. This does not model real-life entropy. Entropy is expensive; if it was not, there would be no need for a DRBG to begin with! Requesting too many bits of entropy when there is insufficient supply can cause a failure. To capture this behavior, I chose to model entropy as an infinite stream of \lstinline{option bool}.

Note that in Coq, as well as many typed programming languages, an \lstinline{option} type is a type that can simply not exist (\lstinline{null} in the language of C and Java). It is effectively equivalent to this:

\begin{lstlisting}
    Inductive option (T: Type) :=
    | Some: T -> option T
    | None: option T.
\end{lstlisting}

A \lstinline{None} value means that the entropy bit is not yet available, and so the call to get entropy should fail. An infinite stream of entropy can be defined as follows:

\begin{lstlisting}
    Definition stream: Type := nat -> option bool.
\end{lstlisting}

The \lstinline{nat} (natural number) argument is used to index into the stream. Note that a simple \lstinline{list} would not work here, because we need the stream to have infinite size.

Any functions that consume entropy will need to return a new entropy stream. Furthermore, because calls to get entropy can fail, such functions have to be able to handle failures. For this reason, I defined a general \lstinline{result} datatype:

\begin{lstlisting}
    Inductive error_code: Type :=
    | catastrophic_error
    | generic_error.

    Inductive result (X: Type): Type :=
    | success: X -> stream -> result X
    | error : error_code -> stream -> result X.
\end{lstlisting}

I also defined a \lstinline{get_bits} function that, if possible, get the consecutive requested number of bits from the stream and returns a new stream that advances the index by the requested number of bits. If \lstinline{get_bits} encounters an error (a \lstinline{None} value), it returns a new stream that skips this error the next time it is called, symbolizing a movement forward in time resulting in more entropy being available.

This stream representation of entropy is a specialized case of the \lstinline{Get_entropy_input} function from \stdtitle{NIST SP 800-90A}, with each bit of the stream representing a bit of entropy, therefore setting \lstinline{min_entropy = min_length = max_length}, with prediction resistance always on. In the language of \stdtitle{NIST SP 800-90A}, this entropy representation is an “ideal random bitstring”.

\section{DRBG Algorithm}
The DRBG algorithm contains four public functions: \lstinline{instantiate}, \lstinline{reseed}, \lstinline{generate}, and \lstinline{uninstantiate}. I will focus on the first three, as \lstinline{uninstantiate} simply involves freeing memory.

\subsection{The DRBG State}

\stdtitle{NIST SP 800-90A \S 10.1.2.1} describes the HMAC DRBG state. In the functional specification, I chose to represent the HMAC DRBG state as the following, which closely resembles the NIST specifications.

\begin{lstlisting}
    Definition DRBG_working_state :=
      (list Z * list Z * Z). (* value * key * reseed_counter *)
    Definition DRBG_state_handle :=
      (DRBG_working_state * Z * bool). (* state, security_strength, prediction_resistance_flag *)
\end{lstlisting}

The two most important fields here are \lstinline{key} and \lstinline{value}. The \lstinline{key} is the secret internal state, used in conjunction with HMAC, while the \lstinline{value} is the “output” of the DRBG. I will explain what “output” means when I explain the \lstinline{generate} function (\S \ref{funcgenerate}).

The distinction between a “working state” and a “state handle” is \stdtitle{NIST SP 800-90A} presents each of the function as first doing error checking then calling an inner algorithm. The error checking part is common to all types of DRBG, not just the HMAC-based one. The inner algorithm is specific to each type of DRBG, and the algorithm only needs the information stored in the working state (and the working state type does not need to be the same for each type of DRBG). See \stdtitle{NIST SP 800-90A \S 8.3} for more details.

Note that \lstinline{Z} is Coq’s representation of arbitrary-precision integers, and is used here to represent a byte. That is, the value of the integer must be between $0$ and $255$, inclusive. I could have represented each byte as a tuple of a \lstinline{Z} and a proof that the value of that \lstinline{Z} is in the byte range, keeping that fact separate allowed me to write a functional specification that closely resembles \stdtitle{NIST SP 800-90A}, and the fact that if the input state contains bytes, then the output will also contain bytes can easily be provided as separate lemmas about the individual functions.

\subsection{The Update Function}
Before showing any of the public function, I must first define a helper function, the \lstinline{HMAC_DRBG_Update} function,  or the \lstinline{update} function. See \stdtitle{NIST SP 800-90A \S 10.1.2.2} for the corresponding NIST specification. The \lstinline{update} function takes a \lstinline{key}, a \lstinline{value}, and an optional nonce and returns a new \lstinline{key} and a new \lstinline{value}. This function will be called at least once in each of the public functions, and the goal is to provide backtracking resistance. What this means is that the original \lstinline{key} cannot be computed from the new \lstinline{key} (note that the \lstinline{value}(s) is public anyways; see \ref{funcgenerate}).

\begin{lstlisting}
    Definition HMAC_DRBG_update (HMAC: list Z -> list Z -> list Z) (provided_data K V: list Z): (list Z * list Z) :=
      let K := HMAC (V ++ [0] ++ provided_data) K in
      let V := HMAC V K in
      match provided_data with
        | [] => (K, V)
        | _::_ =>
          let K := HMAC (V ++ [1] ++ provided_data) K in
          let V := HMAC V K in
          (K, V)
      end.
\end{lstlisting}

Here, \lstinline{provided_data} is the aforementioned optional nonce.

Notice that the backtracking resistance property follows directly from the property of HMAC (that the key cannot be computed).

\subsection{The Instantiate Function}

The \lstinline{instantiate} function initializes the DRBG state with randomness from a source of entropy. The caller is responsible for specifying the minimum security strength for this DRBG, which for an HMAC-based DRBG translates to “how many bits of entropy should be requested?”. In addition to entropy, the DRBG also asks for a nonce, which is roughly defined as a string of bits with at least \lstinline{1/2 security_strength} bits of entropy. In fact, this nonce can even be taken form the same source of entropy. For the purposes of this project, I am choosing to specialize the specification so that the nonce is \textit{always} taken from the same source of entropy. This is compatible with \stdtitle{mbedTLS}.

Here is the functional specification I wrote for the \lstinline{instantiate} function (See \stdtitle{NIST SP 800-90A \S 9.1} for the corresponding NIST specification):

\begin{lstlisting}
    Definition DRBG_instantiate_function
      (instantiate_algorithm: list Z -> list Z -> list Z -> Z -> DRBG_working_state)
      (min_entropy_length max_entropy_length: Z)
      (get_nonce: unit -> list Z)
      (highest_supported_security_strength: Z)
      (max_personalization_string_length: Z)
      (prediction_resistance_supported: bool)
      (entropy_stream: ENTROPY.stream)
      (requested_instantiation_security_strength: Z)
      (prediction_resistance_flag: bool)
      (personalization_string: list Z)
      : ENTROPY.result DRBG_state_handle :=
      if requested_instantiation_security_strength >? highest_supported_security_strength
        then ENTROPY.error ENTROPY.generic_error entropy_stream
      else match prediction_resistance_flag, prediction_resistance_supported with
             | true, false => ENTROPY.error ENTROPY.generic_error entropy_stream
             | _,_ =>
               if (Zlength personalization_string) >? max_personalization_string_length
                 then ENTROPY.error ENTROPY.generic_error entropy_stream
               else
                 let security_strength := if requested_instantiation_security_strength <=? 14 then Some 14
                                      else if requested_instantiation_security_strength <=? 16 then Some 16
                                      else if requested_instantiation_security_strength <=? 24 then Some 24
                                      else if requested_instantiation_security_strength <=? 32 then Some 32
                                      else None in
                 match security_strength with
                   | None => ENTROPY.error ENTROPY.generic_error entropy_stream
                   | Some security_strength =>
                   match (get_entropy security_strength min_entropy_length max_entropy_length
                         prediction_resistance_flag entropy_stream) with
                     | ENTROPY.error e s' => ENTROPY.error ENTROPY.catastrophic_error s'
                     | ENTROPY.success entropy_input entropy_stream =>
                       match (get_entropy (security_strength/2) (min_entropy_length/2)
                             (max_entropy_length/2) prediction_resistance_flag entropy_stream) with
                         | ENTROPY.error e s' => ENTROPY.error ENTROPY.catastrophic_error s'
                         | ENTROPY.success nonce entropy_stream =>
                           let initial_working_state := instantiate_algorithm entropy_input nonce
                             personalization_string security_strength in
                           ENTROPY.success (initial_working_state, security_strength,
                             prediction_resistance_flag) entropy_stream
                       end
                   end
                 end
           end.
\end{lstlisting}

Note that this is just the “outer” function that does error checking on the parameters, as mentioned earlier. Again, this is purely generic; different types of DRBG can share this piece of code. The actual algorithm for computing an initial working state is left as a dependency in the first parameter. All parameters before \lstinline{entropy_stream} are implementation specific constants. For example, \lstinline{prediction_resistance_supported} is a flag denoting whether or not this particular implementation can provide prediction resistance.

Of the actual parameters of this function (everything from and including \lstinline{entropy_stream}), \lstinline{security_strength} has already been explained. \lstinline{prediction_resistance_flag} is used to indicate whether this particular instantiation of DRBG will provide prediction resistance. \lstinline{personalization_string} is expected to be unique, and can include things like the device identifier, timestamp, etc. The idea is not to provide more randomness but to ensure that this particular instantiation will be different from any other instantiations.

The HMAC-based algorithm for instantiation is as follows (See \stdtitle{NIST SP 800-90A \S 10.1.2.3} for the corresponding NIST specification):

\begin{lstlisting}
    Definition HMAC_DRBG_instantiate_algorithm (HMAC: list Z -> list Z -> list Z)
      (entropy_input nonce personalization_string: list Z) (security_strength: Z)
      : DRBG_working_state :=
      let seed_material := entropy_input ++ nonce ++ personalization_string in
      let key := initial_key in
      let value := initial_value in
      let (key, value) := HMAC_DRBG_update HMAC seed_material key value in
      let reseed_counter := 1 in
      (value, key, reseed_counter).
\end{lstlisting}


\lstinline{initial_key} and \lstinline{initial_value} are globally defined constants of length equal to the length of the HMAC output. \lstinline{initial_key} is a string of all byte $0$, while \lstinline{initial_value} is a string of all byte $1$.

I chose to leave the actual HMAC function as a parameter, to allow different implementations to be swapped in (the NIST specification works for all approved HMAC functions).

\subsection{The Reseed Function}
There are times when the DRBG state must be reseeded with new bits of entropy. This can happen because too many bytes have been generated from the current bits of entropy, or because prediction resistance is requested. Reseeding with fresh entropy automatically provides prediction resistance, as the state at time $t + 1$ cannot be computed from the state at time $t$ alone - entropy is also needed. Note that reseeding is preferred over simply instantiating a new DRBG state because if the source of entropy fails silently (supposed the bits of entropy returned are not actually random), at least the DRBG still has some randomness left over from the old entropy.

The functional specification for the “outer” reseed function follows. Again, remember that this is the part that does error checking and is common among all types of DRBG. See \stdtitle{NIST SP 800-90A \S 9.2} for the corresponding NIST specification.

\begin{lstlisting}
    Definition DRBG_reseed_function
      (reseed_algorithm: DRBG_working_state -> list Z -> list Z -> DRBG_working_state)
      (min_entropy_length max_entropy_length: Z)
      (max_additional_input_length: Z)
      (entropy_stream: ENTROPY.stream)
      (state_handle: DRBG_state_handle)
      (prediction_resistance_request: bool)
      (additional_input: list Z)
      : ENTROPY.result DRBG_state_handle :=
      match state_handle with (working_state, security_strength, prediction_resistance_flag) =>
      if prediction_resistance_request && (negb prediction_resistance_flag)
        then ENTROPY.error ENTROPY.generic_error entropy_stream
      else
        if (Zlength additional_input) >? max_additional_input_length
          then ENTROPY.error ENTROPY.generic_error entropy_stream
        else
          match (get_entropy security_strength min_entropy_length max_entropy_length
                prediction_resistance_flag entropy_stream) with
            | ENTROPY.error _ s => ENTROPY.error ENTROPY.catastrophic_error s
            | ENTROPY.success entropy_input entropy_stream =>
              let new_working_state := reseed_algorithm working_state entropy_input additional_input in
              ENTROPY.success (new_working_state, security_strength,
                prediction_resistance_flag) entropy_stream
          end
      end.
\end{lstlisting}

As with \lstinline{instantiate}, the parameters up to but excluding \lstinline{entropy_stream} are implementation specific constants.

Note that both the \lstinline{reseed} and \lstinline{generate} functions allow users to pass in an additional stream of bytes. Users do not need to provide this (and implementations can choose to omit this feature), but it is meant to provide any additional bits of entropy or to provide personalization of this specific call.

The HMAC-based reseed algorithm follows (See \stdtitle{NIST SP 800-90A \S 10.1.2.4} for the corresponding NIST specification):

\begin{lstlisting}
    Definition HMAC_DRBG_reseed_algorithm (HMAC: list Z -> list Z -> list Z)
      (working_state: DRBG_working_state) (entropy_input additional_input: list Z)
      : DRBG_working_state :=
      match working_state with (v, key, _) =>
        let seed_material := entropy_input ++ additional_input in
        let (key, v) := HMAC_DRBG_update HMAC seed_material key v in
        let reseed_counter := 1 in
        (v, key, reseed_counter)
      end.
\end{lstlisting}

\subsection{The Generate Function} \label{funcgenerate}
The \lstinline{generate} function is used to generate an arbitrary number of pseudorandom bytes, automatically calling \lstinline{reseed} first if necessary. A reseed is necessary if the caller asks for prediction resistance or if the inner (HMAC-based) \lstinline{generate} algorithm returns a flag saying a reseed is necessary. Note that the DRBG state has to have been initialized with prediction resistance capability if the caller wants prediction resistance.

The outer function body for \lstinline{generate} follows. See \stdtitle{NIST SP 800-90A \S 9.3} for the corresponding NIST specification. Unlike with the previous functions, a recursive helper function is needed to specify this function body because of the use of a \lstinline{goto} statement in \stdtitle{NIST SP 800-90A}. The recursive helper function captures the “reseed if required” logic.

\begin{lstlisting}
    Fixpoint DRBG_generate_function_helper
      (generate_algorithm: DRBG_working_state -> Z -> list Z -> DRBG_generate_algorithm_result)
      (reseed_function: ENTROPY.stream -> DRBG_state_handle -> bool -> list Z
        -> ENTROPY.result DRBG_state_handle)
      (entropy_stream: ENTROPY.stream)
      (state_handle: DRBG_state_handle)
      (requested_number_of_bytes: Z)
      (prediction_resistance_request: bool)
      (additional_input: list Z)
      (should_reseed: bool)
      (count: nat)
      : ENTROPY.result (list Z * DRBG_working_state) :=
      let result := if should_reseed then
                      match (reseed_function entropy_stream state_handle prediction_resistance_request
                        additional_input) with
                        | ENTROPY.success x entropy_stream => ENTROPY.success (x, []) entropy_stream
                        | ENTROPY.error e entropy_stream => ENTROPY.error e entropy_stream
                      end
                    else ENTROPY.success (state_handle, additional_input) entropy_stream in
      match result with
        | ENTROPY.error e s => ENTROPY.error e s
        | ENTROPY.success (state_handle, additional_input) entropy_stream =>
          match state_handle with (working_state, security_strength, prediction_resistance_flag) =>
            match generate_algorithm working_state requested_number_of_bytes additional_input with
              | generate_algorithm_reseed_required =>
                match count with
                  | O => ENTROPY.error ENTROPY.generic_error entropy_stream (* impossible *)
                  | S count' => (DRBG_generate_function_helper generate_algorithm reseed_function
                    entropy_stream state_handle requested_number_of_bytes prediction_resistance_request
                    additional_input true count')
                end
              | generate_algorithm_success x y => ENTROPY.success (x, y) entropy_stream
            end
          end
        end.

    Definition DRBG_generate_function
      (generate_algorithm: Z -> DRBG_working_state -> Z -> list Z -> DRBG_generate_algorithm_result)
      (reseed_function: ENTROPY.stream -> DRBG_state_handle -> bool -> list Z
        -> ENTROPY.result DRBG_state_handle)
      (reseed_interval: Z)
      (max_number_of_bytes_per_request: Z)
      (max_additional_input_length: Z)
      (entropy_stream: ENTROPY.stream)
      (state_handle: DRBG_state_handle)
      (requested_number_of_bytes requested_security_strength: Z)
      (prediction_resistance_request: bool)
      (additional_input: list Z)
      : ENTROPY.result (list Z * DRBG_state_handle) :=
      match state_handle with (working_state, security_strength, prediction_resistance_flag) =>
        if requested_number_of_bytes >? max_number_of_bytes_per_request
          then ENTROPY.error ENTROPY.generic_error entropy_stream
        else
          if requested_security_strength >? security_strength
            then ENTROPY.error ENTROPY.generic_error entropy_stream
          else
            if (Zlength additional_input) >? max_additional_input_length
              then ENTROPY.error ENTROPY.generic_error entropy_stream
            else
              if prediction_resistance_request && (negb prediction_resistance_flag)
                then ENTROPY.error ENTROPY.generic_error entropy_stream
              else
                match (DRBG_generate_function_helper (generate_algorithm reseed_interval)
                reseed_function entropy_stream state_handle requested_number_of_bytes
                prediction_resistance_request additional_input prediction_resistance_request 1%nat) with
                  | ENTROPY.error e s => ENTROPY.error e s
                  | ENTROPY.success (output, new_working_state) entropy_stream =>
                      ENTROPY.success (output, (new_working_state, security_strength,
                        prediction_resistance_flag)) entropy_stream
                end
      end.
\end{lstlisting}


Note that \lstinline{Fixpoint} is a recursive function where at least one parameter is structurally getting smaller. In this case, it is the last parameter, the natural number \lstinline{count}, which is initialized with \lstinline{1}, meaning that this recursive function is only allowed to go one level deep.

As usual, the \lstinline{generate} algorithm is left as a dependency. Also left as a dependency is the \lstinline{reseed} function. The signature matches the \lstinline{reseed} function shown earlier, albeit with the dependencies and implementation constants filled in.

The HMAC-based \lstinline{generate} algorithm follows. See \stdtitle{NIST SP 800-90A \S 10.1.2.5} for the corresponding NIST specification. A recursive function is also needed to specify the algorithm, due to the use of a \lstinline{while} loop in \stdtitle{NIST SP 800-90A}.


\begin{lstlisting}
    Function HMAC_DRBG_generate_helper_Z (HMAC: list Z -> list Z -> list Z) (key v: list Z)
      (requested_number_of_bytes: Z) {measure Z.to_nat requested_number_of_bytes}: (list Z * list Z) :=
        if 0 >=? requested_number_of_bytes then (v, [])
        else
          let len := 32%nat in
          let (v, rest) := HMAC_DRBG_generate_helper_Z HMAC key v
            (requested_number_of_bytes - (Z.of_nat len)) in
          let v := HMAC v key in
          let temp := v in
          (v, rest ++ temp).

    Definition HMAC_DRBG_generate_algorithm (HMAC: list Z -> list Z -> list Z) (reseed_interval: Z)
      (working_state: DRBG_working_state) (requested_number_of_bytes: Z) (additional_input: list Z)
      : DRBG_generate_algorithm_result :=
        match working_state with (v, key, reseed_counter) =>
          if reseed_counter >? reseed_interval then generate_algorithm_reseed_required
          else
            let (key, v) := match additional_input with
                              | [] => (key, v)
                              | _::_ => HMAC_DRBG_update HMAC additional_input key v
                            end in
            let (v, temp) := HMAC_DRBG_generate_helper_Z HMAC key v requested_number_of_bytes in
            let returned_bits := firstn (Z.to_nat requested_number_of_bytes) temp in
            let (key, v) := HMAC_DRBG_update HMAC additional_input key v in
            let reseed_counter := reseed_counter + 1 in
            generate_algorithm_success returned_bits (v, key, reseed_counter)
        end.
\end{lstlisting}

A \lstinline{Function} is the general form of a recursive function, and unlike \lstinline{Fixpoint}, typically requires a proof that the function does terminate. In this case, the argument \lstinline{requested_number_of_bytes} is decreasing, but in a nonobvious way (not structurally). The proof is simple, but long (about as long as the function body). Luckily, it is machine-checked, so I will not present it here.

The \lstinline{generate} algorithm first checks if the number of times \lstinline{generate} has been called (\lstinline{reseed_counter}) is too high, and if so, returns a flag saying a reseed is needed. Otherwise, it repeatedly HMACs the current \lstinline{value} with the current \lstinline{key}, append the result to the output, and store the result as the new \lstinline{value}. This is why the \lstinline{value} part of the DRBG state is not considered secret, and is really the output of the DRBG. The output is then trimmed down to the requested number of bytes, and \lstinline{reseed_counter} is incremented by \lstinline{1}.

\section{Preparation Instructions}

\subsection{Paper Formatting}

There are no minimum or maximum length limits on IW reports.
We are including this template because we think it will be helpful
for citing things properly and for including figures into formatted
text.  If you are using \LaTeX~\cite{lamport94}
to typeset your paper, then we strongly suggest
that you start from the template available at
http://iw.cs.princeton.edu -- this
document was prepared with that template.
If you are using a different software package to typeset your paper,
then you can still use this document as a reasonable sample of
how your report might look.  Table~\ref{table:formatting} is a suggestion
of some formatting guidelines, as well as being an example of how to
include a table in a Latex document.

\begin{table}[h!]
  \centering
  \begin{tabular}{|l|l|}
    \hline
    \textbf{Field} & \textbf{Value}\\
    \hline
    \hline
    Paper size & US Letter 8.5in $\times$ 11in\\
    \hline
    Top margin & 1in\\
    \hline
    Bottom margin & 1in\\
    \hline
    Left margin & 1in\\
    \hline
    Right margin & 1in\\
    \hline
    Body font & 12pt\\
    \hline
    Abstract font & 12pt, italicized\\
    \hline
    Section heading font & 14pt, bold\\
    \hline
    Subsection heading font & 12pt, bold\\
    \hline
  \end{tabular}
  \caption{Formatting guidelines. }
  \label{table:formatting}
\end{table}

\textbf{Please ensure that you include page numbers with your
submission}. This makes it easier for readers to refer to
different parts of your paper when they provide comments.

We highly recommend you use bibtex for managing your references and citations.  You can add bib entries to a references.bib file throughout the semester (e.g. as you read papers) and then they will be ready for you to cite when you start writing the report.  If you use bibtex, please note that the references.bib file provided in the template example includes some format-specific incantations at the top of the file.  If you substitute your own bib file, you will probably want to include these
incantations at the top of it.

\subsection{Citations}

There are various reasons to cite prior work and include it as references in your bibliography.  For example, If you are improving upon
prior work, you should include
a full citation for the work in the bibliography \cite{nicepaper,nicepaper2}.
You can also cite information that is used as background or explanation.  In addition to citing scholarly papers or books, you can also create bibtex entries for webpages or other sources.  Many online databases allow you to download a premade bibtex entry for each paper you access.  You can simply copy-paste these into your references.bib file.

\noindent\textbf{Figures and Tables.} Ensure that the figures and
tables are legible.  Please also ensure that you refer to your
figures in the main text. Make sure that your figures will be legible
in the expected forms that the report will be read.  If you expect someone
to print it out in gray-scale, then make sure the figures are legible
when printed that way.

\noindent\textbf{Main Body.} Avoid bad page or column breaks in
your main text, i.e., last line of a paragraph at the top of a
column or first line of a paragraph at the end of a column. If you
begin a new section or sub-section near the end of a column,
ensure that you have at least 2 lines of body text on the same
column.

\subsection{Ethics}

Your independent work report should abide by the basic standards of scholarly ethics and by the Princeton Honor Code. If you have any doubts about how to cite
other work, how to quote or include text or images from other works, or other issues, please discuss them with your project adviser or with the IW coordinators.

\section{Outline}
The following is a possible outline for your paper.
\subsection{Introduction}
\begin{itemize}
\item Problem statement
\item Motivation and goal...The goal of this project is...
\item Roadmap: The remainder of this paper is organized as follows....
\end{itemize}

\subsection{Related Work}
\begin{itemize}
\item Survey of prior work with similar goals
\item Comparison to your project
\end{itemize}

\subsection{Approach}
\begin{itemize}
\item Key novel idea
\end{itemize}

\subsection{Implementation}
\begin{itemize}
\item Things you implemented.  How you did it?
\end{itemize}

\subsection{Evaluation}
\begin{itemize}
\item Experiment design...
\item Data...
\item Metrics...
\item Comparisons...
\item Qualitative results...
\item Quantitative results...
\item Further results needed...
\end{itemize}

\subsection{Summary}
\begin{itemize}
\item Conclusions...
\item Limitations...
\item Future work...
\end{itemize}

\cite{sha}
\cite{hmac}
\cite{prg}
\cite{nist}

\bstctlcite{bstctl:etal, bstctl:nodash, bstctl:simpurl}
\bibliographystyle{IEEEtranS}
\bibliography{references}


\end{document}
