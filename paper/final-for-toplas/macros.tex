
%%%%%%%%%%%%%%%%%%%%%%%%%%%%%%%%%%%%%%%%%%%%%%%%%%%%%%%%%%%%%%%%%%%%%%%
%% Rustan wants the letters in his math formulas to be spaced
%% as in running italic text.  So we here redefine each letter,
%% in its role as a MathSymbol, to come from the symbol font
%% named "italics", rather than from the one named "letters".

% The following line is needed to make this file work with certain
% fonts, include the standard Computer Modern fonts.
\DeclareSymbolFont{italics}{OT1}{cmr}{m}{it}

\DeclareMathSymbol{a}{\mathalpha}{italics}{"61}
\DeclareMathSymbol{b}{\mathalpha}{italics}{"62}
\DeclareMathSymbol{c}{\mathalpha}{italics}{"63}
\DeclareMathSymbol{d}{\mathalpha}{italics}{"64}
\DeclareMathSymbol{e}{\mathalpha}{italics}{"65}
\DeclareMathSymbol{f}{\mathalpha}{italics}{"66}
\DeclareMathSymbol{g}{\mathalpha}{italics}{"67}
\DeclareMathSymbol{h}{\mathalpha}{italics}{"68}
\DeclareMathSymbol{i}{\mathalpha}{italics}{"69}
\DeclareMathSymbol{j}{\mathalpha}{italics}{"6A}
\DeclareMathSymbol{k}{\mathalpha}{italics}{"6B}
\DeclareMathSymbol{l}{\mathalpha}{italics}{"6C}
\DeclareMathSymbol{m}{\mathalpha}{italics}{"6D}
\DeclareMathSymbol{n}{\mathalpha}{italics}{"6E}
\DeclareMathSymbol{o}{\mathalpha}{italics}{"6F}
\DeclareMathSymbol{p}{\mathalpha}{italics}{"70}
\DeclareMathSymbol{q}{\mathalpha}{italics}{"71}
\DeclareMathSymbol{r}{\mathalpha}{italics}{"72}
\DeclareMathSymbol{s}{\mathalpha}{italics}{"73}
\DeclareMathSymbol{t}{\mathalpha}{italics}{"74}
\DeclareMathSymbol{u}{\mathalpha}{italics}{"75}
\DeclareMathSymbol{v}{\mathalpha}{italics}{"76}
\DeclareMathSymbol{w}{\mathalpha}{italics}{"77}
\DeclareMathSymbol{x}{\mathalpha}{italics}{"78}
\DeclareMathSymbol{y}{\mathalpha}{italics}{"79}
\DeclareMathSymbol{z}{\mathalpha}{italics}{"7A}

\DeclareMathSymbol{A}{\mathalpha}{italics}{"41}
\DeclareMathSymbol{B}{\mathalpha}{italics}{"42}
\DeclareMathSymbol{C}{\mathalpha}{italics}{"43}
\DeclareMathSymbol{D}{\mathalpha}{italics}{"44}
\DeclareMathSymbol{E}{\mathalpha}{italics}{"45}
\DeclareMathSymbol{F}{\mathalpha}{italics}{"46}
\DeclareMathSymbol{G}{\mathalpha}{italics}{"47}
\DeclareMathSymbol{H}{\mathalpha}{italics}{"48}
\DeclareMathSymbol{I}{\mathalpha}{italics}{"49}
\DeclareMathSymbol{J}{\mathalpha}{italics}{"4A}
\DeclareMathSymbol{K}{\mathalpha}{italics}{"4B}
\DeclareMathSymbol{L}{\mathalpha}{italics}{"4C}
\DeclareMathSymbol{M}{\mathalpha}{italics}{"4D}
\DeclareMathSymbol{N}{\mathalpha}{italics}{"4E}
\DeclareMathSymbol{O}{\mathalpha}{italics}{"4F}
\DeclareMathSymbol{P}{\mathalpha}{italics}{"50}
\DeclareMathSymbol{Q}{\mathalpha}{italics}{"51}
\DeclareMathSymbol{R}{\mathalpha}{italics}{"52}
\DeclareMathSymbol{S}{\mathalpha}{italics}{"53}
\DeclareMathSymbol{T}{\mathalpha}{italics}{"54}
\DeclareMathSymbol{U}{\mathalpha}{italics}{"55}
\DeclareMathSymbol{V}{\mathalpha}{italics}{"56}
\DeclareMathSymbol{W}{\mathalpha}{italics}{"57}
\DeclareMathSymbol{X}{\mathalpha}{italics}{"58}
\DeclareMathSymbol{Y}{\mathalpha}{italics}{"59}
\DeclareMathSymbol{Z}{\mathalpha}{italics}{"5A}
%% end Rustan's math fonts

%-----------------compact itemization ------------------------------------
\newenvironment{ditemize}{%   itemized lists w/o interline space
\begin{list}{$\bullet$}{%
\setlength{\itemsep}{0pt}\setlength{\rightmargin}{0pt}%
\setlength{\leftmargin}{1em}\setlength{\parsep}{0ex}}}{
\end{list}}
%-------------------------------------------------------------------------

% GS macros

\newcommand{\bigsepcon}{\overlay{\bigodot}{*}}
\newcommand{\ccoresem}[4]{(#1, #2, #3, #4)} %compcert coresem
\newcommand{\ecoresem}[4]{(\mathbb{#1}, \mathbb{#2}, \mathbb{#3}, \mathbb{#4})} %extended coresem
\renewcommand{\to}{\rightarrow}
\newcommand{\topart}{\rightharpoonup}
\newcommand{\defeq}{\triangleq}

\newcommand{\injects}[3]{#2 \rightarrowtail_{#1} #3}

\newcommand{\upd}[3]{#1[#2\mapsto #3]}
\newcommand{\eval}[6]{#1 \vdash #5 \Downarrow_{#2, #3, #4} #6}
\newcommand{\step}[5]{
  #1 \vdash \langle #2, #3\rangle \longmapsto \langle #4, #5\rangle
}
\newcommand{\twostep}[5]{
  #1 \vdash \!\!\!\begin{array}{l}
               \langle #2, #3\rangle \longmapsto \\ 
               \langle #4, #5\rangle
            \end{array}
}
\newcommand{\csemstep}[6]{
  #1 \vdash_{#6} \langle #2, #3\rangle \longmapsto \langle #4, #5\rangle
}
\newcommand{\twolinestep}[5]{
  \begin{array}{rl} #1 \vdash\!\!\!\!\!\! & \langle #2, #3\rangle \longmapsto \\ 
                              & \langle #4, #5\rangle 
  \end{array}
}
\newcommand{\twolinestepplus}[5]{
  \begin{array}{rl} #1 \vdash\!\!\!\!\!\! & \langle #2, #3\rangle \longmapsto^{+} \\ 
                              & \langle #4, #5\rangle 
  \end{array}
}
\newcommand{\stepplus}[5]{
  #1 \vdash \langle #2, #3\rangle \longmapsto^{+} \langle #4, #5\rangle
}
\newcommand{\stepstar}[5]{
  #1 \vdash \langle #2, #3\rangle \longmapsto^{*} \langle #4, #5\rangle
}

%%% Syntax
\newcommand{\keyw}[1]{\mathsf{#1}}
\newcommand{\option}[1]{\keyw{option}\;#1}
\newcommand{\List}[1]{\keyw{list}\;#1}
\newcommand{\Type}{\keyw{Type}}
\newcommand{\Prop}{\mathbb{T}}
\newcommand{\Some}[1]{\keyw{Some}\ #1}

\newcommand{\state}[3]{\keyw{RunState}(#1, #2, #3)}

\newcommand{\fsim}[2]{#1 \preceq #2}
\newcommand{\match}[6]{
  \langle #3, #4\rangle \sim_{#2; #1} \langle #5, #6\rangle
}
\newcommand{\matchnoo}[5]{
  \langle #2, #3\rangle \sim_{#1} \langle #4, #5\rangle
}
\newcommand{\matchi}[7]{
  \langle #4, #5\rangle \sim^{#1}_{#3; #2} \langle #6, #7\rangle
}

%types
\newcommand{\valtype}{\mathcal{V}}
\newcommand{\eftype}{\mathcal{F}}

\newcommand{\sequent}[4]{#1 \land \,#2\ \vdash\ #3 \land \,#4}
\newcommand{\simplesequent}[2]{#1\ \vdash\ #2}

\newcommand{\triple}[3]{\{#1\}\,#2\,\{#3\}}
