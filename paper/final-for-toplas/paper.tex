%  one-way hash function

\documentclass[prodmode,acmtoplas]{acmsmall}
%\documentclass[prodmode,acmtoplas]{acmsmallhacked}

\newcommand\blind[2]{#2}

% correct bad hyphenation here
\hyphenation{Comp-Cert}
\hyphenation{Comp-Cert-TSO}

%\usepackage{amsmath,amssymb,infrule,stmaryrd,mathtools,array}
\usepackage{amsmath,infrule}
%\usepackage[pdftex]{graphicx}
%\usepackage{multicol}
%\usepackage{wrapfig}
%\usepackage{tikz}
\usepackage{soul}
%\usetikzlibrary{decorations.pathmorphing}
%\usepackage{enumerate}
%\usetikzlibrary{matrix,arrows}
%\usepackage{times}
%\usepackage{adjustbox}

\usepackage{listings}
\usepackage{lstlangcoq}
\lstset{language=Coq,basicstyle=\sffamily,mathescape=true,columns=fullflexible}


%%%%%%%%%%%%%%%%%%%%%%%%%%%%%%%%%%%%%%%%%%%%%%%%%%%%%%%%%%%%%%%%%%%%%%%
%% Rustan wants the letters in his math formulas to be spaced
%% as in running italic text.  So we here redefine each letter,
%% in its role as a MathSymbol, to come from the symbol font
%% named "italics", rather than from the one named "letters".

% The following line is needed to make this file work with certain
% fonts, include the standard Computer Modern fonts.
\DeclareSymbolFont{italics}{OT1}{cmr}{m}{it}

\DeclareMathSymbol{a}{\mathalpha}{italics}{"61}
\DeclareMathSymbol{b}{\mathalpha}{italics}{"62}
\DeclareMathSymbol{c}{\mathalpha}{italics}{"63}
\DeclareMathSymbol{d}{\mathalpha}{italics}{"64}
\DeclareMathSymbol{e}{\mathalpha}{italics}{"65}
\DeclareMathSymbol{f}{\mathalpha}{italics}{"66}
\DeclareMathSymbol{g}{\mathalpha}{italics}{"67}
\DeclareMathSymbol{h}{\mathalpha}{italics}{"68}
\DeclareMathSymbol{i}{\mathalpha}{italics}{"69}
\DeclareMathSymbol{j}{\mathalpha}{italics}{"6A}
\DeclareMathSymbol{k}{\mathalpha}{italics}{"6B}
\DeclareMathSymbol{l}{\mathalpha}{italics}{"6C}
\DeclareMathSymbol{m}{\mathalpha}{italics}{"6D}
\DeclareMathSymbol{n}{\mathalpha}{italics}{"6E}
\DeclareMathSymbol{o}{\mathalpha}{italics}{"6F}
\DeclareMathSymbol{p}{\mathalpha}{italics}{"70}
\DeclareMathSymbol{q}{\mathalpha}{italics}{"71}
\DeclareMathSymbol{r}{\mathalpha}{italics}{"72}
\DeclareMathSymbol{s}{\mathalpha}{italics}{"73}
\DeclareMathSymbol{t}{\mathalpha}{italics}{"74}
\DeclareMathSymbol{u}{\mathalpha}{italics}{"75}
\DeclareMathSymbol{v}{\mathalpha}{italics}{"76}
\DeclareMathSymbol{w}{\mathalpha}{italics}{"77}
\DeclareMathSymbol{x}{\mathalpha}{italics}{"78}
\DeclareMathSymbol{y}{\mathalpha}{italics}{"79}
\DeclareMathSymbol{z}{\mathalpha}{italics}{"7A}

\DeclareMathSymbol{A}{\mathalpha}{italics}{"41}
\DeclareMathSymbol{B}{\mathalpha}{italics}{"42}
\DeclareMathSymbol{C}{\mathalpha}{italics}{"43}
\DeclareMathSymbol{D}{\mathalpha}{italics}{"44}
\DeclareMathSymbol{E}{\mathalpha}{italics}{"45}
\DeclareMathSymbol{F}{\mathalpha}{italics}{"46}
\DeclareMathSymbol{G}{\mathalpha}{italics}{"47}
\DeclareMathSymbol{H}{\mathalpha}{italics}{"48}
\DeclareMathSymbol{I}{\mathalpha}{italics}{"49}
\DeclareMathSymbol{J}{\mathalpha}{italics}{"4A}
\DeclareMathSymbol{K}{\mathalpha}{italics}{"4B}
\DeclareMathSymbol{L}{\mathalpha}{italics}{"4C}
\DeclareMathSymbol{M}{\mathalpha}{italics}{"4D}
\DeclareMathSymbol{N}{\mathalpha}{italics}{"4E}
\DeclareMathSymbol{O}{\mathalpha}{italics}{"4F}
\DeclareMathSymbol{P}{\mathalpha}{italics}{"50}
\DeclareMathSymbol{Q}{\mathalpha}{italics}{"51}
\DeclareMathSymbol{R}{\mathalpha}{italics}{"52}
\DeclareMathSymbol{S}{\mathalpha}{italics}{"53}
\DeclareMathSymbol{T}{\mathalpha}{italics}{"54}
\DeclareMathSymbol{U}{\mathalpha}{italics}{"55}
\DeclareMathSymbol{V}{\mathalpha}{italics}{"56}
\DeclareMathSymbol{W}{\mathalpha}{italics}{"57}
\DeclareMathSymbol{X}{\mathalpha}{italics}{"58}
\DeclareMathSymbol{Y}{\mathalpha}{italics}{"59}
\DeclareMathSymbol{Z}{\mathalpha}{italics}{"5A}
%% end Rustan's math fonts

%-----------------compact itemization ------------------------------------
\newenvironment{ditemize}{%   itemized lists w/o interline space
\begin{list}{$\bullet$}{%
\setlength{\itemsep}{0pt}\setlength{\rightmargin}{0pt}%
\setlength{\leftmargin}{1em}\setlength{\parsep}{0ex}}}{
\end{list}}
%-------------------------------------------------------------------------

% GS macros

\newcommand{\bigsepcon}{\overlay{\bigodot}{*}}
\newcommand{\ccoresem}[4]{(#1, #2, #3, #4)} %compcert coresem
\newcommand{\ecoresem}[4]{(\mathbb{#1}, \mathbb{#2}, \mathbb{#3}, \mathbb{#4})} %extended coresem
\renewcommand{\to}{\rightarrow}
\newcommand{\topart}{\rightharpoonup}
\newcommand{\defeq}{\triangleq}

\newcommand{\injects}[3]{#2 \rightarrowtail_{#1} #3}

\newcommand{\upd}[3]{#1[#2\mapsto #3]}
\newcommand{\eval}[6]{#1 \vdash #5 \Downarrow_{#2, #3, #4} #6}
\newcommand{\step}[5]{
  #1 \vdash \langle #2, #3\rangle \longmapsto \langle #4, #5\rangle
}
\newcommand{\twostep}[5]{
  #1 \vdash \!\!\!\begin{array}{l}
               \langle #2, #3\rangle \longmapsto \\ 
               \langle #4, #5\rangle
            \end{array}
}
\newcommand{\csemstep}[6]{
  #1 \vdash_{#6} \langle #2, #3\rangle \longmapsto \langle #4, #5\rangle
}
\newcommand{\twolinestep}[5]{
  \begin{array}{rl} #1 \vdash\!\!\!\!\!\! & \langle #2, #3\rangle \longmapsto \\ 
                              & \langle #4, #5\rangle 
  \end{array}
}
\newcommand{\twolinestepplus}[5]{
  \begin{array}{rl} #1 \vdash\!\!\!\!\!\! & \langle #2, #3\rangle \longmapsto^{+} \\ 
                              & \langle #4, #5\rangle 
  \end{array}
}
\newcommand{\stepplus}[5]{
  #1 \vdash \langle #2, #3\rangle \longmapsto^{+} \langle #4, #5\rangle
}
\newcommand{\stepstar}[5]{
  #1 \vdash \langle #2, #3\rangle \longmapsto^{*} \langle #4, #5\rangle
}

%%% Syntax
\newcommand{\keyw}[1]{\mathsf{#1}}
\newcommand{\option}[1]{\keyw{option}\;#1}
\newcommand{\List}[1]{\keyw{list}\;#1}
\newcommand{\Type}{\keyw{Type}}
\newcommand{\Prop}{\mathbb{T}}
\newcommand{\Some}[1]{\keyw{Some}\ #1}

\newcommand{\state}[3]{\keyw{RunState}(#1, #2, #3)}

\newcommand{\fsim}[2]{#1 \preceq #2}
\newcommand{\match}[6]{
  \langle #3, #4\rangle \sim_{#2; #1} \langle #5, #6\rangle
}
\newcommand{\matchnoo}[5]{
  \langle #2, #3\rangle \sim_{#1} \langle #4, #5\rangle
}
\newcommand{\matchi}[7]{
  \langle #4, #5\rangle \sim^{#1}_{#3; #2} \langle #6, #7\rangle
}

%types
\newcommand{\valtype}{\mathcal{V}}
\newcommand{\eftype}{\mathcal{F}}

\newcommand{\sequent}[4]{#1 \land \,#2\ \vdash\ #3 \land \,#4}
\newcommand{\simplesequent}[2]{#1\ \vdash\ #2}

\newcommand{\triple}[3]{\{#1\}\,#2\,\{#3\}}


\begin{document}

%\titlebanner{\textbf{DRAFT---please do not distribute.}}        % These are ignored unless
%\preprintfooter{CompCert simulations}   % 'preprint' option specified.

%
% paper title
% can use linebreaks \\ within to get better formatting as desired
\markboth{Andrew W. Appel}{Verification of a Cryptographic Primitive: SHA-256}
\title{Verification of a Cryptographic Primitive: SHA-256}
\author{ANDREW W. APPEL \affil{Princeton University}}

\begin{bottomstuff}
This material is based on research sponsored by the DARPA under
agreement number FA8750-12-2-0293.  The U.S. Government is authorized
to reproduce and distribute reprints for Governmental purposes
notwithstanding any copyright notation thereon. The views and
conclusions contained herein are those of the authors and should not
be interpreted as necessarily representing the official policies or
endorsements, either expressed or implied, of DARPA or the
U.S. Government.
\end{bottomstuff}



\begin{abstract}
A full formal machine-checked verification of a C program:
the OpenSSL implementation of SHA-256.  This is an interactive
proof of functional correctness
in the Coq proof assistant, using the Verifiable C program logic.
Verifiable C is a separation logic for the C language,
proved sound w.r.t. the operational semantics for C,
connected to the CompCert verified optimizing C compiler.
\end{abstract}

\category{D.2.4}{Software/Program Verification}{Correctness proofs}
\category{E.3}{Data Encryption}{Standards}
\category{F.3.1}{Specifying and Verifying and Reasoning about Programs}{}
\terms{Verification}

% \keywords keyword1, keyword2

%\newtheorem{lemma}{Lemma}
%\newtheorem{theorem}{Theorem}

\maketitle

\section{Introduction}
\label{sec:intro}

\begin{quote}
\itshape
\qquad [C]ryptography is hard to do right, and the only way to know if something was done right is to be able to examine it.\,\ldots
This argues very strongly for open source cryptographic algorithms.\,\ldots 
[But] simply publishing the code does not automatically mean that people will examine it for security flaws. \hfill \hbox{Bruce Schneier \citeyear{schneier99:open}}

\qquad Be suspicious of commercial encryption software \ldots 
\linebreak[2] [because of]
back doors.\,\ldots
Try to use public-domain encryption that has to be compatible with
other implementations.\,\ldots'' \hfill \hbox{Bruce Schneier \citeyear{schneier13:how}}
\end{quote}
That is, use widely used, well examined open-source implementations
of published, nonproprietary, widely used, well examined,
standard algorithms---because ``many eyes make all bugs shallow''
works only if there are many eyes paying attention.

To this I add: use implementations that are 
\emph{formally verified with machine-checked proofs}
of functional correctness, of side-channel resistance,
of information-flow properties.
``Many eyes'' are a fine thing, but sometimes it takes
them a couple of years to notice the bugs \cite{heartbleed14}.
Verification can guarantee program properties
in advance of widespread release.

In this paper I present a first step:
a formal verification of the functional
correctness of the SHA-256 implementation
from the OpenSSL open-source distribution.

Formal verification is not necessarily a 
\emph{substitute} for many-eyes assurance.
For example, in this case, I present only 
the assurance of functional correctness
(and its corollary, safety, including
absence of buffer overruns).
With respect to other properties such
as timing side channels, I prove nothing;
so it is comforting that this same C
program has over a decade of widespread
use and examination.

SHA-256, the Secure Hash Algorithm with 256-bit digests,
is not an encryption algorithm, but it is used in
encryption protocols.  The methods I discuss in this
paper can be applied to the same issues that appear
in ciphers such as AES: interpretation of standards
documents, big-endian protocols implemented on little-endian
machines, odd corners of the C semantics, 
storing bytes and loading words,
signed and unsigned arithmetic, 
extended precision arithmetic,
trustworthiness of C compilers, 
use of machine-dependent special instructions to make things faster,
correspondence of models to programs,
assessing the trusted base of the verification tools.

This paper presents the following result:
I have proved functional correctness
of the OpenSSL implementation of SHA-256,
with respect to a \emph{functional 
specification:} a formalization of the FIPS 180-4
\emph{Secure Hash Standard} \cite{fips180-4}.
The machine-checked proof is done using the 
\emph{Verifiable C} program logic,
in the Coq proof assistant.
Verifiable C is proved sound with respect to
the operational semantics of C,
with a machine-checked proof in Coq.
The C program can be compiled to x86
assembly language with the
CompCert verified optimizing C compiler;
that compiler is proved correct (in Coq) with respect
to the same operational semantics of C
and the semantics of x86 assembly language.
Thus, by composition of machine-checked proofs
with no gaps, the assembly-language program
correctly implements the functional specification.

In addition, I implemented SHA-256 as a
functional program in Coq and proved it
equivalent to the functional specification.
Coq can execute the functional program
on real strings
(only a million times slower than the C program), and gets
the same answer as standard reference implementations.\footnote{
That's 0.25 seconds per block, versus 0.25 microseconds;
fast enough for testing the specification.
The Coq functional program is a million times slower because it simulates the
logical theory of binary integers used in the specification!  
The functional spec is \emph{even slower than that,} because its
$W$ function takes a factor of $4^{16}$ more time.}
This gives some extra confidence that no
silly things are wrong with the functional spec.

\paragraph{Limitations}
The implementation is from OpenSSL, with some
macro expansion to instantiate it
from generic SHA-2 to SHA-256.
I factor assignment statements so that
there is at most one memory operand per command,
e.g., \lstinline[language=C]{ctx->h[0] += a;} becomes
\lstinline[language=C]{t=ctx->h[0]; ctx->h[0]=t+a;}
see~\S\ref{sec:changes}.

CompCert generates assembly language, not
machine language; there is no correctness
proof of the assembler or of the 
x86 processor hardware 
on which one might run the compiled program.
The Coq proof assistant is widely used,
and its kernel is believed to be a correct
implementation of the Predicative Calculus of Inductive
Constructions (CiC), which in turn
is believed to be consistent.

A different kind of limitation is in the
time and cost of doing the verification.
SHA-256 was the ``shakedown cruise'' for
the \emph{Verifiable C} system.  This
``cruise'' revealed many inefficiencies
of Verifiable C's proof automation system:
it is slow, it is a memory hog
and it is difficult to use in places,
and it is \emph{incomplete:} some corners
of the C language have inadequate 
automation support.  But this incompleteness,
or shakiness of proof automation,
cannot compromise the
end-to-end guarantee of machine-checked
logical correctness: every proof step
is checked by the Coq kernel.

\paragraph{Nonlimitations}
The other way that my implementation
differs from OpenSSL is that I used
the x86's byte-swap instruction
in connection with 
big-endian 4-byte load/store 
(since it is a little-endian machine).
This illustrates a common practice 
when implementing cryptographic
primitives on general-purpose microprocessors:
use machine-dependent special instructions
to gain performance.  It is good that
the program logic can reason about such
instructions.

What about using \texttt{gcc} or LLVM to compile
SHA-256?  Fortunately, these compilers
(\texttt{gcc}, LLVM, CompCert) agree quite well
on the C semantics, so a verification
of SHA-256 can still add assurance for users of 
other C compilers. In most of the rare places they disagree,
CompCert is correct and the others are exhibiting a bug
\cite{yang2012finding}; no bugs have ever been found
in the phases of CompCert behind Verifiable C.\footnote{
That is, CompCert has a front-end phase from C to C light;
Verifiable C plugs in \emph{after} this phase, at C light.
\citeN{yang2012finding} found a bug or two
in that front-end phase at a time when that
phase was not formally verified, but they could not
find \emph{any} bugs in any of the verified phases,
the ones between C light and assembly language.
Since then, Leroy
has formally verified the C-to-Clight phase,
but that doesn't matter for Verifiable C,
because in effect we verify functional correctness
of the C light program.
Also, \citeN{yang2012finding}
found a \emph{specification} bug in CompCert, regarding
how it treated bit-fields.  Although Leroy has since
fixed that specification bug, this also does not
matter:  Verifiable C is immune to specification
bugs in C, for reasons discussed in \S\ref{sec:specright}.}
\section{Verified Software Toolchain}

The Verified Software Toolchain (VST) \cite{appel14:plcc}
contains the \emph{Verifiable C} program logic for the C language,
proved sound with respect to the operational
semantics of CompCert C.  The VST has
proof automation tools for applying this program logic
to C programs.

One style of formal verification of software proceeds
by applying a \emph{program logic}
to a program.  An example of a program logic
is Hoare logic, which relates the program
$c$ to its specification (precondition
$P$, postcondition $Q$) via the 
judgment $\{P\} c \{Q\}$.  This 
\emph{Hoare triple} can be proved by 
using the inference rules of the Hoare logic,
such as the \emph{sequential composition} rule:

\infrule[]{\triple{P}{c_1}{Q}\andalso \triple{Q}{c_2}{R}}
{\triple{P}{c_1;c_2}{R}}

We prefer \emph{sound} program logics
or analysis algorithms, i.e., where there
is a proof that whatever the program logic claims
about your program is actually true when the program
executes. The VST is proved sound by giving a semantic
model of the Hoare judgment with respect to the
full operational semantics of CompCert C, so that
we can really say, \emph{what you prove in Verifiable
C is what you get when the source program executes.}
CompCert is itself proved correct, so we can say,
\emph{what you get when the source program executes
is the same when the compiled program executes.}
Composing these three proofs together: 
the proof of a program, the soundness of Verifiable C,
and the correctness of CompCert,
we get: \emph{the compiled program satisfies its 
specification.}

C programs are tricky to verify because one needs to keep
track of many side conditions and restrictions:
this variable \emph{is} initialized here,
that addition \emph{does not} overflow,
this $p<q$ compares pointers into \emph{the same}
object, that pointer \emph{is not} dangling.
The Verifiable C logic keeps track of every one of these;
or rather, assists the user in keeping track of every one.
We know that there are no missed assumptions,
because of the soundness proof w.r.t. the C semantics;
and we know the C semantics does not miss any,
because of the CompCert correctness proof w.r.t.
safe executability in assembly language.

Of course, there are easier ways to prove programs
correct.  One can write functional programs in languages
(such as Gallina, ML, Haskell) 
with much cleaner proof theories than C, and then 
the proof effort is smaller by an order of magnitude.
Whenever the performance of a high-level garbage-collected
language is tolerable, this is the way to go.  The
vast amount of software that is today written in Perl, Python,
Javascript might profitably be rewritten in functional
languages with clean proof theories for effective verification.
But cryptographic primitives are not written in these languages;
if we want to verify a well established widely used open-source
cryptographic implementation, we need tooling for C.

\paragraph{Synthesis instead of verification?}
C was not designed with a simple proof theory in mind,
so perhaps a simpler route to verified crypto would be
to use \emph{program synthesis} from a domain-specific
specification language.
One example is Cryptol \cite{erkok2009hardware} which can generate either C or
VHDL directly from a functional specification.
In principle one could hope to prove the Cryptol 
synthesizer correct (though this has not been done) or validate the output
(which might be easier than proving general-purpose C programs).

Unfortunately, synthesis languages 
sometimes have limited expressiveness.
Cryptol has been used to synthesize the block-shuffle part of SHA
from a functional spec---but only the block-shuffle
(the function that OpenSSL calls 
\textsf{sha256\_block\_data\_order}).\footnote{
Aaron Tomb, Galois.com, personal communication, 13 January 2014.}
Using Verifiable C I have verified the
entire implementation, including the padding, length
computation, multi-block handling, incremental update
of unaligned strings, and so on.  
The Cryptol synthesizers (which translate Cryptol to C or VHDL)
can handle only fixed-size blocks, so cannot handle these
parts (\textsf{SHA256\_Init,}
\textsf{SHA256\_Update,}
\textsf{SHA256\_Final,}
\textsf{SHA256}).

\section{Specification of SHA-256}
\label{functional-spec}

A program without a specification \emph{cannot be incorrect,} it can
only be surprising.\footnote{Paraphrase of J. J. Horning, 1982.}
Typically one might prove a C program correct with respect
to a \emph{relational specification}.
For example, a C implementation implementing lookup tables
must satisfy this relation between program states
and inputs/outputs:
that if the most recent binding for $x$ is
$x\mapsto y$, then looking up $x$ yields $y$.

Sometimes one does this in two stages:
prove that the C program correctly
implements a \emph{functional specification}
(an abstraction of the implementation),
then prove that functional specification
satisfies the relational specification.
For example, a C implementation implementing lookup tables by balanced
binary search trees might be proved correct with respect to a
functional-spec of red-black trees.  
Then the functional red-black trees can (more easily)
be proved to have the lookup-table property.

For cryptographic hashing, we%
\blind{}{\footnote{Stephen Yi-Hsien Lin wrote a functional spec of SHA-256
in Coq, which I subsequently adapted and rewrote.}}
built a functional
spec from the FIPS 180-4 standard \cite{fips180-4}.  The relational
spec is, ``implements a random function.''  Unfortunately,
nobody in the world knows how to prove that SHA-256 implements
a random function---even on paper---so I did not attempt
a machine-checked proof of that (see \S\ref{context}).

The FIPS 180-4 SHS (Secure Hash Standard) 
mentions (in \S 3.2) 32-bit unsigned
binary arithmetic with operators such as addition-modulo-32,
exclusive-or, and shifting.  We must give a model of 32-bit
arithmetic in pure logic.  Fortunately, Leroy has defined
such an \textsf{Integers} module and proved many of its properties
as part of the semantics of CompCert C \cite{leroy09};
we use this directly in the functional spec, which is
otherwise entirely independent of the C language.
We have: the type \textsf{int};
operations such as \textsf{Int.add},
\textsf{Int.xor};
injection $(\mathsf{Int.repr}: \mathbf{Z}\rightarrow \mathsf{int})$ from the mathematical integers
to 32-bit integers, and projection 
$(\mathsf{Int.unsigned}: \mathsf{int}\rightarrow \mathbf{Z})$.
We have ``axioms'' such as,
\infrule{0 \le i < 2^{32}}
{\mathsf{Int.unsigned}~(\mathsf{Int.repr}~i)~=~i}
but this is not an axiom of the underlying logic (CiC),
it is a theorem proved\footnote{Henceforth, ``proved'' can be understood
to mean, ``proved with a machine-checked proof in Coq.''}
from the axioms of Coq using the constructive definitions of 
\textsf{Int.unsigned} and \textsf{Int.repr}.

SHS defines SHA-256 on a bitstring of any length,
and explains how to pack these into big-endian 32-bit integers.
OpenSSL's implementation permits any sequence of
bytes, that is, multiples of 8 bits.  We represent
a sequence of byte values by a sequence of mathematical
integers, and we can define the big-endian packing function as,
\begin{lstlisting}
Definition Z_to_Int (a b c d : Z) : Int.int :=
  Int.or (Int.or (Int.or (Int.shl (Int.repr a) (Int.repr 24)) (Int.shl (Int.repr b) (Int.repr 16)))
    (Int.shl (Int.repr c) (Int.repr 8)))      (Int.repr d).
\end{lstlisting}
Given a list \lstinline{nl} of byte values (represented
as mathematical integers, type \lstinline{Z}),
if the length of \lstinline{nl} is a multiple of
4, it's simple to define the corresponding
list of big-endian 32-bit integers:
\begin{lstlisting}
Fixpoint Zlist_to_intlist(nl: list Z): list int :=
  match nl with h1::h2::h3::h4::t => Z_to_Int h1 h2 h3 h4 :: Zlist_to_intlist t
$\hspace{65pt}$  | _ => nil
  end.
\end{lstlisting}
Coq uses \textsf{Definition} for a nonrecursive function (or value,
or type), and \textsf{Fixpoint} for structurally recursive functions.
The operator \textsf{::} is list \emph{cons.}  The 
operations such as \lstinline{Int.shl} (shift left)
and \lstinline{Int.repr} (the 32-bit representation of a mathematical
integer) are given foundational meaning by Leroy's 
\lstinline{Int} package, as explained above.

SHS defines several functions \emph{Ch, Maj, ROTR, SHR, 
$\Sigma_0^{\{256\}}$, $\Sigma_1^{\{256\}}$ $\sigma_0^{\{256\}}$, $\sigma_1^{\{256\}}$},
for example,
\[
\begin{array}{rcl}
\mathsf{Ch}(x,y,z) &=& (x\wedge y)\oplus(\neg x \wedge z) \\
\mathsf{Maj}(x,y,z) &=& (x\wedge y)\oplus(x \wedge z)\oplus (y \wedge z) 
\end{array}\]
Translating these into Coq is quite straightforward:
\begin{lstlisting}
Definition Ch (x y z : int) : int :=  Int.xor (Int.and x y) (Int.and (Int.not x) z).
Definition Maj (x y z : int) : int :=  Int.xor (Int.xor(Int.and x z)(Int.and y z))(Int.and x y).
Definition Rotr b x : int := Int.ror x (Int.repr b).
Definition Shr b x : int := Int.shru x (Int.repr b).
Definition Sigma_0 (x : int) : int :=  Int.xor (Int.xor (Rotr 2 x) (Rotr 13 x)) (Rotr 22 x).
Definition Sigma_1 (x : int) : int := ...
Definition sigma_0 (x : int) : int := ...
Definition sigma_1 (x : int) : int := ...
\end{lstlisting}

The vector $K_0^{\{256\}}\ldots K_{63}^{\{256\}}$
is given as a series of 32-bit hexadecimal constants,
as is the vector $H_0^{(0)}\ldots H_7^{(0)}$.
In Coq we write them in decimal, and inject with 
\lstinline{Int.repr}:
\begin{lstlisting}
Definition K := map Int.repr [1116352408 , 1899447441, 3049323471,  ..., 3329325298].

Definition initial_registers := Map Int.repr [1779033703, 3144134277, ..., 1541459225].
\end{lstlisting}

Given a message $M$ of length $\mathcal{\ell}$ bits,
the SHS explains: append a 1 bit, then enough zero bits
so the length-appended message will be a multiple of the
block size,
then a 64-bit representation of the length.
Since we have a message $M$ of length $n$ bytes;
we append a 128 byte (which already has 7 trailing zeros),
then the appropraiate number of zero bytes.
We big-endian convert this to 32-bit integers,
then add two more 32-bit integers representing the
high-order and low-order parts of the length-in-bits.
\begin{lstlisting}
Definition generate_and_pad M := 
 let n := Zlength M in
 Zlist_to_intlist (M ++ [128%Z] ++ list_repeat (Z.to_nat (-(n + 9) mod 64)) 0)
$\hspace{1in}$   ++ [Int.repr (n * 8 / Int.modulus), Int.repr (n * 8)].
\end{lstlisting}
Note that $0 \le a \bmod 64 < 64$ even if $a$ is negative.
The magic number 9 comes from 1+8: 1 terminator byte (value
128) plus 8 bytes for the 64-bit length field.
Taking $-(n+9)$ mod 64 gives the number of bytes of
padding necessary to round up to the next multiple of 64
bytes, which is the block size.

SHS defines the message schedule $W_t$ as follows:
\[
W_t = \left\{ \begin{array}{ll} M_t^{(i)} & ~~0 \le t \le 15 \\[1ex]
\sigma_1^{\{256\}}(W_{t-2}) + W_{t-7} +
\sigma_0^{\{256\}}(W_{t-15}) + W_{t-16}
      & ~~16 \le t \le 63
\end{array}
\right.
\]
where the superscript $(i)$ indicates
the value in the $i$th message block.
We translate this into Coq as,
\begin{lstlisting}
Function W (M: Z -> int) (t: Z) {measure Z.to_nat t} : int :=
  if zlt t 16 
  then M t 
  else  (Int.add (Int.add (sigma_1 (W M (t-2))) (W M (t-7)))
          (Int.add (sigma_0 (W M (t-15))) (W M (t-16)))).
Proof.
intros; apply Z2Nat.inj_lt; omega. (* t-2 < t *)
intros; apply Z2Nat.inj_lt; omega. (* t-7 < t *)
intros; apply Z2Nat.inj_lt; omega. (* t-15 < t *)
intros; apply Z2Nat.inj_lt; omega. (* t-16 < t *)
Qed.
\end{lstlisting}
Coq is a language of total functions.  The \textsf{measure}
and \textsf{Proof/Qed} demonstrate that the
$W$ function always terminates.  There is one
proof line for each of the 4 recursive calls;
each  proof is, ``calling on a smaller value 
of $t$.''

One could run this $W$ as a functional program;
but it takes time exponential in $t$,
since there are 4 recursive calls.
It serves well as a functional spec
but it is not practically executable.

The block cipher computes 256-bit
(8-word) hashes of 512-bit (16-word)
blocks.  The accumulated hash
of the first $i$ blocks is
the vector $H_0^{(i)}\ldots H_7^{(i)}$.
To hash the next block,
the eight ``working variables'' $a$
through $h$ are initialized
from the $H^{(i)}$ vector.
Then 64 iterations of this \texttt{Round} function
are executed:
\begin{lstlisting}
Definition registers := list int.

Function Round  (regs: registers) (M: Z ->int) (t: Z) 
     {measure (fun t => Z.to_nat(t+1)) t} : registers :=
 if zlt t 0 then regs 
 else match Round regs M (t-1) with
  | [a,b,c,d,e,f,g,h] => 
    let T1 := Int.add(Int.add(Int.add(Int.add h (Sigma_1 e))(Ch e f g))(nthi K t))(W M t) in
    let T2 := Int.add (Sigma_0 a) (Maj a b c) in 
     [Int.add T1 T2, a, b, c, Int.add d T1, e, f, g]
  | _ => nil
  end.
Proof. intros; apply Z2Nat.inj_lt; omega. Qed.
\end{lstlisting}
That is, one calls \lstinline{(Round r (nthi block) 63)}.
(The function $\mathsf{nthi}~b~i$ returns the $i$th element
of the list $b$, or the arbitrary element \lstinline{Int.zero}
if $i$ is negative or beyond the length of the list.)
If the length of the \textsf{regs} list is not 8,
an arbitrary result (the empty list) is returned;
but it \emph{will} be 8.

I represent \lstinline{registers} as a list, rather
than a dependently typed vector (i.e., a list whose
type inherently enforces the length restriction)
to keep the specification as first-order as possible.
This simplifies reasoning about the specification,
especially its portability to other logics.

The round function returns registers
$a',b',\ldots,h'$ which are then added 
to the $H^{(i)}$ to yield $H^{(i+1)}$:

\begin{lstlisting}
Definition hash_block (r: registers) (block: list int) : registers :=
      map2 Int.add r (Round r (nthi block) 63).
\end{lstlisting}

Given a message of length $16k$, the following function
computes the $H^{(k)}$ by applying 
\lstinline{hash_block} to each successive 
16-word block:
\begin{lstlisting}
Function hash_blocks (r: registers) (msg: list int) {measure length msg} : registers :=
  match msg with
  | nil => r
  | _ => hash_blocks (hash_block r (firstn 16 msg)) (skipn 16 msg)
  end.
Proof. ... Qed.
\end{lstlisting}

Finally, the \lstinline{SHA_256} produces the message digest
as a 32-byte string by the big-endian conversion of $H^{(k)}$.

\begin{lstlisting}
Definition SHA_256 (str : list Z) : list Z :=
    intlist_to_Zlist (hash_blocks init_registers (generate_and_pad str)).
\end{lstlisting}

\section{Functional program}
One can prove that a program satisfies a specification,
but how does one know that the specification 
is properly written down?  One way to gain confidence
in the specification is to calculate its results
on a series of examples, i.e., to run it.
The \lstinline{SHA_256} function given above
is actually an \emph{executable specification}.
Coq permits relational (nonconstructive, propositional) 
specifications that do not run, but also permits
fully constructive specifications such as this one.

However, since the $W$ function is exponential,
it's impractical to run this program.  
Therefore I wrote an alternative functional program
called \lstinline{SHA_256'}.
One can run this program directly inside the Coq 
proof assistant on actual inputs; it takes about
0.25 seconds per block.

The key to this ``efficiency'' is that
the \lstinline{W} function should remember
its previous results.\footnote{Also in this efficient program the
\lstinline{generate_and_pad} function is done quite differently.}
Here \lstinline{msg}
is the \emph{reversed} list \linebreak[3]
$W_{t-1},W_{t-2},W_{t-3},...W_0$:
\begin{lstlisting}
Definition Wnext (msg : list int) : int :=
 match msg with
 | x1::x2::x3::x4::x5::x6::x7::x8::x9::x10::x11::x12::x13::x14::x15::x16::_ =>
     (Int.add (Int.add (sigma_1 x2) x7) (Int.add (sigma_0 x15) x16))
 | _ $$ => Int.zero  (* impossible *)
 end.
\end{lstlisting}
Should we be worried about the ``impossible'' case?
Coq is a language of total functions, so we must return \emph{something}
here. One reason we need not worry is that
I \emph{proved} that this case cannot cause the
\lstinline{SHA_256'} program to be wrong.
That is, I proved the equivalence (using the extensionality axiom):
\begin{lstlisting}
Lemma SHA_256'_eq:  SHA_256' = SHA_256.
\end{lstlisting}

Also, the fact that \lstinline{SHA_256'}
gives the right answer---on all the inputs that
I tried---allows us to \emph{know} that
\lstinline{SHA_256} is also right on those inputs.

This equivalence proof took about a day to build;
this is much faster than building the proof
that the C implementation correctly implements
the functional spec.  But sometimes we must
program in C---to get SHA that runs
in microseconds rather than seconds.

Instead of calculating the result inside Coq,
one could instead extract the program as an
ML program, and compile with the OCaml compiler.
This would lead to faster results than
Coq, but slower than C.
For the purpose of testing the SHA specification,
it is unnecessary.

\section{Introduction to Verifiable C}

The \emph{Verifiable C} language and program logic
is a subset of CompCert's \emph{C Light} 
language.  Every Verifiable C
program is a legal C program,
and every C program can be expressed in Verifiable C
with only local program transformations,
such as pulling side effects out of expressions:
\lstinline{a=(b+=2)+3;}
becomes 
\lstinline{b+=2; a=b+3;}
(sometimes an extra local variable is required 
to hold the intermediate result).
The CompCert compiler accepts (essentially)
the full C language; we use a subset
not because of any inadequacy of CompCert
but to accommodate reasoning about the program.
It is easier to reason about one assignment
at a time.

\paragraph{Separation logic}
We use a variant of  Hoare logic
known as \emph{separation logic},
which is more expressive regarding
anti-aliasing of pointers and separation
of data structures.
We write $\triple{P}c{Q}$
to mean (more or less), if $P$ holds
before $c$ executes, then $Q$ will hold after.
In our separation logic, the assertion $P$
has a \emph{spatial} part dealing with
the contents of \emph{a particular footprint}
of memory, and a \emph{local} part dealing with
local program variables, and a \emph{propositional}
part dealing with mathematical variables.
An example of a \emph{spatial} assertion is,
\[
\mathsf{array}_{[0,n)}^f(p) 
\qquad\qquad
(n:\mathbf{Z},~f:\mathbf{Z}\rightarrow \mathbf{V},~p:\mathbf{V})
\]
which represents an array of $n$ elements starting at address $p$,
whose $i$th element is $f(i)$.  Here $f$ is a total function from
(mathematical) integers to values; we ignore $f$'s domain outside
 $[0,n)$.  Values $\mathbf{V}$ 
may be 32-bit integers (\lstinline{Vint $i$}),
32-bit representations of mathematical integers (\lstinline{Vint (Int.repr $z$)}),
floating point (\lstinline{Vfloat $f$}), pointers (\lstinline{Vptr $b$ $i$}) with base $b$
and in-the-block offset $i$, or undefined/uninitialized values 
(\lstinline{Vundef}).

Verifiable C's array constructor actually takes two more arguments:
a permission share $\pi$ indicating read-only, read-write, etc.; 
and the C-language type of the elements, such as the type of
unsigned characters,
\begin{lstlisting}
Definition tuchar := Tint I8 Unsigned noattr.
\end{lstlisting}

Suppose we have two different arrays $p,q$ and we execute
the assignment \textsf{p[i]=q[j];} one possible
specification is this:
\[
\begin{array}{l}
\{ 0\le i < j < n \,\wedge\,(\mathsf{array}_{[0,n)}^f(p)\,*\,\mathsf{array}_{[0,n)}^g(q)) \} \\
\mathtt{t=q[j]; \quad p[i]=t;} \\
\{ 0\le i < j < n \,\wedge\,(\mathsf{array}_{[0,n)}^{f[i:=g(j)]}(p)\,*\,\mathsf{array}_{[0,n)}^g(q)) \}
\end{array}
\]
Separation logic's inference rules prefer reasoning about
one load or store at a time, so I have made a local program
transformation.
Here I assume there are two disjoint arrays $p$ and $q$ 
whose contents are $f$ and $g$ respectively.  You can tell they are
disjoint because the $*$ operator enforces this.
Because they are disjoint, we know (in the postcondition) that
$q$ is unchanged, i.e., its contents are still $g$.
(If the programmer had intended $p$ and $q$ to possibly overlap,
one would write a different specification.)

\paragraph{Program variables, symbolic values}
This is a bit of a simplification: $i,j,p,q$ are program variables,
not logical variables.  Verifiable C distinguishes these; one might
write the precondition ``for real'' as,
\begin{lstlisting}
PROP ($0\le i<j<n$; writable_share $\pi_1$)
LOCAL(`(eq $i$) (eval_id _i); `(eq $j$) (eval_id _j); `(eq $p$) (eval_id _p); `(eq $q$) (eval_id _q))
SEP  (`(array_at tuchar $\pi_1$ $f$ 0 $n$ $p$); $~$ `(array_at tuchar $\pi_2$ $g$ 0 $n$ $q$))
\end{lstlisting}
where the \lstinline{PROP} part has
pure logical propositions (that do not refer to 
program state);
\lstinline{LOCAL} gives assertions about
local variables of the program state (but not memory);
and \lstinline{SEP} is the
\emph{sep}arating conjunction of \emph{spatial} assertions,
i.e., about various disjoint parts of memory.

The notation \lstinline{`(eq $i$) (eval_id _i)}
means, ``C program variable \lstinline{_i}
contains the symbolic value $i$.
This is effectively a statement about the current program state's
local-variable environment $\rho$.
The notation \lstinline{`$f$} lifts $f$ over
local-variable environments \cite[Chapter 21]{appel14:plcc},
that is,
\begin{lstlisting}
`(eq $i$) (eval_id _i) $~~$ = $~~$ (fun $\rho$ => (eq $i$) (eval_id _i $\rho$)) $~~$ =
(fun $\rho$ => (fun $x$ => $i=x$) (eval_id _i $\rho$)) $~~$ = $~~$ (fun $\rho$ => $i$ = eval_id _i $\rho$).
\end{lstlisting}
or in other words, looking up \lstinline{_i} in $\rho$ yields $i$.

\paragraph{Permissions}
The $p$ array needs to be writable, while the $q$ array
needs to be at least read-only.  This is expressed
with permission-shares:  $p$'s permission-share
$\pi_1$ needs to satisfy the \lstinline{writable_share}
predicate.  We don't need to say
\lstinline{readable_share $\pi_2$} because that is
implied by the \lstinline{array_at} predicate.

We call these \emph{permission shares} rather than just 
\emph{permissions} because in the shared-memory concurrent
setting, a proof could split $\pi_1=\pi_{1a} + \pi_{1b}$ into smaller shares
that are given to concurrent threads.  These shares
$\pi_{1a}$ and $\pi_{1b}$ would not be strong enough for write permission,
but they could be both strong enough for read permission.
That permits exclusive-write-concurrent-read protocols.
Now, suppose SHA-256 were called in one thread of a concurrent
program.  Its parameter (the string to be hashed) could be a
read-only shared array, but its result (the array to hold
the message digest) must be writable.  All this is 
concisely expressed in the permission-share annotation of my
SHA-256 specification.

\paragraph{Control flow}
The C language has control flow: a command $c$ might fall-through normally,
might \texttt{continue} a loop, or \texttt{break} a loop,
or \texttt{return} from a function.  Thus the postcondition $Q$
must have up to four different assertions for these cases.
For the case where all but fall-through are prohibited---i.e., three of these four postcondition-assertions are \textbf{False}---use
the construction \lstinline{normal_ret_assert}.
\begin{lstlisting}
normal_ret_assert (
 PROP ()
 LOCAL(`(eq $i$) (eval_id _i); `(eq $j$) (eval_id _j); `(eq $p$) (eval_id _p); `(eq $q$) (eval_id _q))
 SEP  (`(array_at tuchar $\pi_1$ (upd $f$ $i$ $(q\,j)$) 0 $n$ $p$); $~$ `(array_at tuchar $\pi_2$ $g$ 0 $n$ $q$)))
\end{lstlisting}
This postcondition
shows that the $p$ array has changed in one spot
and $q$ has not changed.  We can omit $(0\le i<j<n)$
from the postcondition, since it's a logical fact independent
of state, and (if true in the precondition)
is eternally true.

\paragraph{Higher-order reasoning}
Ordinary separation logic is inexpressive regarding
function-pointers, data abstraction, and concurrency;
so Verifiable C is a higher-order impredicative
concurrent separation logic.  
Higher-order means that one can quantify over
predicates.  This is useful for specifying
abstract data types.  It is also useful for
function pointers:  if function $f$
takes a parameter $p$ that's a function-pointer,
then the precondition of $f$ will characterize
the \emph{specification} of $p$, i.e.,
$p$'s precondition and postcondition.
When function-pointer specifications
are used to describe object-oriented
programs, then impredicative
quantification over these
specifications is necessary.

One might think that C is not an object-oriented
language, but in fact C programmers often use
design patterns that they express with \lstinline{void *}.
C's type system is to weak to ``prove'' that all these
\lstinline{void *} casts turn out all right, but we
can specify and prove this with the program logic.

The SHA verification
does not use these higher-order features,
though one could use data abstraction for
the context structure, \lstinline{SHA256state_st}.
However, OpenSSL uses an object-oriented
``engine'' construction to compose HMAC with SHA.

\paragraph{Further reading}
\citeN{appel14:plcc} give a full explanation of
the program logic.

\section{The C program}

The OpenSSL implementation of SHA-256 is clever
in several ways (many of which were 
intentional in the SHA-256 design):
\begin{enumerate}
\item It works in one pass, waiting until the end
before adding the padding and length.
\item It allows incremental hashing.  Suppose
the message to be hashed is available,
sequentially, in segments $s_1,s_2,\ldots,s_j$.
One calls \textsf{SHA256\_Init} to initialize
a context, \textsf{SHA256\_Update} with
each $s_i$ in turn, then
\textsf{SHA256\_Final} to add the padding
and length and hash the last block.
If the $s_i$ are not block-aligned,
then \textsf{SHA256\_Update} remembers
partial blocks in a buffer.  However,
a block internal to one of the $s_i$ is
not cycled through the buffer; the 
\textsf{sha256\_block\_data\_order}
function operates on it directly from the
memory where it was passed to 
\textsf{SHA256\_Update}.
\item Within the 64-round computation,
it store \emph{only} the
the most recent 16 elements of $W_t$, 
in a buffer accessed modulo 16
using bitwise-and in the array subscript.
\item In adding the length of $s_i$ to the
accumulated 64-bit count of bits,
there is an overflow test: is the result
of $(a+b)\mod 2^{32} < a$?  If so,
add a carry to the high-order word.  Such tests
are easy to get wrong \cite{wang2013towards};
here it works because $a,b$ are declared
\textsf{unsigned}, but still a proof is worthwhile.
\item In the \textsf{SHA256\_Final}
function, there is one last block
containing the 1-bit, padding, and length.
But there could be two ``last'' blocks, if the message body
ends within 8 bytes of the end of a block
(so there's no room for the 1-bit plus
64-bit length).
\item The accumulated state between
calls to \textsf{SHA256\_Update} is kept
in a record ``owned'' by the caller
and initialized by \textsf{SHA256\_Init}.
But the $W$ vector is purely local
to the ``round'' function
(\textsf{sha256\_block\_data\_order}),
so is kept as a local-variable
(stack-allocated) array.  Although that's not
\emph{particularly} clever, it's too clever
for some C-language verification systems,
which (unlike Verifiable C)
cannot handle addressable local variables
\cite{greenaway2012bridging,carbonneau14:pldi}.
\end{enumerate}

\noindent The client of SHA-256 calls upon it as follows:

\begin{lstlisting}
typedef struct SHA256state_st {
    unsigned int h[8];  $~~~~~~\mbox{\emph{// The H vector}}$
    unsigned int Nl,Nh;     $~~~\mbox{\emph{// Length, a 64-bit number in two parts}}$
    unsigned char data[64];  $\mbox{\emph{// Partial block not yet hashed}}$
    unsigned int num;   $~~~~~\mbox{\emph{// Length of the message fragment}}$
} SHA256_CTX;

SHA256_CTX $c$;
char $\mathit{digest}$[32];
char *$m_1$, *$m_2$, ..., *$m_k$;
unsigned int $n_1$, $n_2$, ..., $n_k$;

$\mbox{\emph{// How the caller hashes a message:}}$
SHA256_Init(&$c$);
SHA256_Update(&$c$, $m_1$, $n_1$);
SHA256_Update(&$c$, $m_2$, $n_2$);
...
SHA256_Update(&$c$, $m_k$, $n_k$);
SHA256_Final($\mathit{digest}$, &$c$);
\end{lstlisting}
The strings $m_i$ of lengths $n_i$ respectively
make up the message.  The idea is that \lstinline{Init}
sets up the context $c$ with 
the initial register state \lstinline{$c$.h[ ]}, 
and then each Update hashes some more blocks
into that register state.  If $m_i$ is not
a full block, or rather if $\sum_{j=1}^i n_j$ is not
a multiple of the block size, then a partial
block is saved in (copied into) the context $c$.
Then the $(i+1)$th call to \lstinline{Update} will
use that fragment as the beginning of the next full block.
The $m_i$ need not be disjoint; the caller can
build the successive parts of the message in the same $m$ buffer.

After the $i$th call, the registers \lstinline{$c$.h[ ]}
contain the hash of all the full blocks seen so far,
and the length \lstinline{Nl,Nh} contains the
length (in bits) of all the message fragments,
i.e., $8 \cdot \sum_{j=1}^i n_j$.

At the end, the \lstinline{Final} call adds the
padding and length, and hashes the last block(s).
The final \lstinline{$c$.h[ ]} values are then
returned as a byte-string, the message $\mathit{digest}$.

Most of these ``clever'' implementation choices
are not directly visible in the functional specification,
and are not representable in a domain-specific language
such as Cryptol.  They are just general-purpose C programming,
and our specification language must be able to reason
about them.

\section{Specifying the C program}
The Separation Logic specification of a C program
relates the program (and its in-memory data structures)
to functional or relational correctness properties.
Appendix~\ref{sec:the-spec} gives the full separation-logic specification
of the OpenSSL SHA-256 program; 
here I present a part of it.

The \lstinline{SHA256_CTX} data structure
has a \emph{concrete} meaning
and an \emph{abstract} meaning.
The concrete meaning is given by this 6-tuple of values,
corresponding to the 6 fields of the struct:
\begin{lstlisting}
Definition s256state := (list val * (val * (val * (list val * val)))).
             (*comment: $~$ h $~~~~~~~$ Nl $~~~~$ Nh  $~~~~$ data $~~$  num *)
\end{lstlisting}
There's a specific reason for using a tuple here, instead of a Coq
record:  this tuple type is calculated automatically from
the C-language \emph{struct} definition, \emph{inside}
Coq's calculational logic.

The abstract meaning is that
all the full blocks of $m_1+m_2+\ldots+m_i$
have been
parsed\footnote{SHS uses the word ``parsed'' to indicate:
grouping bytes/bits into big-endian 32-bit words,
and grouping 32-bit words into 16-word blocks.}
into a sequence of 32-bit words that we call \lstinline{hashed};
and the remaining less-than-a-block
fragment is a sequence of bytes that we call \lstinline{data}.
\pagebreak
\begin{lstlisting}
Inductive s256abs :=  $~~~~~$(* SHA-256 abstract state *)
 S256abs: forall (hashed: list int) $~~~$(* words hashed, so far *)
            (data: list Z), $~~~~~~~~$ (* bytes in partial block *)
             s256abs.
\end{lstlisting}
This fancy notation is really just a 2-tuple (hashed,data);
I define it this way to influence the names Coq chooses
for introduced variables.

The abstract state is an abstraction of the concrete state.
I make this relation formal in Coq as follows.
First, we calculate what the $H$ vector
would be at the end of \lstinline{hashed}:
\begin{lstlisting}
Definition s256a_regs (a: s256abs) : list int :=
 match a with S256abs hashed data  => hash_blocks init_registers hashed  end.
\end{lstlisting}
Notice that this calls upon \lstinline{hash_blocks}
from the functional spec described in section~\ref{functional-spec}.

Next, we can calculate the bit-length
of the hashed words plus the data bytes:
\begin{lstlisting}
Definition s256a_len (a: s256abs) : Z := 
  match a with S256abs hashed data => (Zlength hashed * 4 + Zlength data) * 8  end.
\end{lstlisting}
We can define the 64-bit concatenation
of two 32-bit numbers,
and what it means for a (mathematical) integer
to be representable in an unsigned char:
\begin{lstlisting}
Definition hilo (hi: int) (lo:int) : Z := (Int.unsigned hi * Int.modulus + Int.unsigned lo).
Definition isbyteZ (i: Z) := (0 $\le$ i < 256).
\end{lstlisting}
Finally, here is the abstraction relation:
\begin{lstlisting}
Definition s256_relate (a: s256abs) (r: s256state) : Prop :=
  match a with S256abs hashed data =>
    s256_h r = map Vint (hash_blocks init_registers hashed) 
  /\ (exists$$hi, exists lo, s256_Nh r = Vint hi /\ s256_Nl r = Vint lo /\
        (Zlength hashed * 4 + Zlength data)*8 = hilo hi lo)
  /\ s256_data r = map Vint (map Int.repr data)
  /\ (length data < 64 /\ Forall isbyteZ data)
  /\ (16 | Zlength hashed)
  /\ s256_num r = Vint (Int.repr (Zlength data))
  end.
\end{lstlisting}
That is, 
a concrete state $(
r_\mathrm{h},
r_\mathrm{Nh},
r_\mathrm{Nl},
r_\mathrm{data},
r_\mathrm{num})$
represents an abstract state $a=(\mathit{hashed},\mathit{data})$
whenever:
\begin{itemize}
\item $r_\mathrm{h}$ is the result of hashing
all of $\mathit{hashed}$;
\item the bit-length of $(\mathit{hashed},\mathit{data})$ equals
$r_\mathrm{Nh}\cdot 2^{32}+r_\mathrm{Nl}$;
\item the sequence of char values $r_\mathrm{data}$
corresponds exactly to the sequence of (mathematical)
integers $\mathit{data}$;
\item the length of $\mathit{data}$ is less than the block
size, and every element of data is $0 \le d < 256$;
\item the length of $\mathit{hashed}$ is a multiple of 16 words;
\item the length of $\mathit{data}$
is $r_\mathrm{num}$ bytes.
\end{itemize}

Verifiable C's logic has an operator 
\lstinline{(data_at $\pi$ $\tau$ $r$ $p$)}
saying that memory-address $p$,
interpreted according to the C-language type $\tau$,
contains (struct/array/integer) data value $r$
with access permission $\pi$.
For example: \lstinline{$\tau=$t_struct_SHA256state_st},
$p$ is a pointer to \lstinline{struct SHA256state_st},
$r$ is a concrete-state value 
$(
r_\mathrm{h},
r_\mathrm{Nh},
r_\mathrm{Nl},
r_\mathrm{data},
r_\mathrm{num})$,
and $\pi$ is the full-access permission \lstinline{Tsh}.

To relate the in-memory \lstinline{SHA256_CTX}
to an abstract state, we simply compose the
relations \lstinline{data_at} and \lstinline{s256_relate}:
\begin{lstlisting}
Definition sha256state_ ($a$: s256abs) ($c$: val) : mpred :=
   EX $r$: s256state, 
    PROP (s256_relate $a$ $r$)
    LOCAL ()
    SEP (data_at Tsh t_struct_SHA256state_st $r$ $c$).
\end{lstlisting}
This relates $a$ to $c$ by saying there exists a concrete state $r$
such that abstract-to-concrete composes with concrete-in-memory.

\paragraph{Incremental update}
The \lstinline{SHA256_Update} function
updates a context $c$ with the bytes \lstinline{data_}
of length \lstinline{len}:
\begin{verbatim}
void SHA256_Update (SHA256_CTX *c, const void *data_, size_t len);
\end{verbatim}
Suppose \lstinline{data_} contains
the sequence of integers $\mathit{msg}$.
Appending $\mathit{msg}$ to an abstract state 
$a=(\mathit{hashed},\mathit{oldfrag})$
yields the updated abstract state 
$a'=(\mathit{hashed}\mathtt{++}\mathit{blocks},\mathit{newfrag})$
when,
\begin{lstlisting}
Inductive update_abs: list Z -> s256abs -> s256abs -> Prop :=
 Update_abs:
   (forall $\mathit{msg}$ $\mathit{hashed}$ $\mathit{blocks}$ $\mathit{oldfrag}$ $\mathit{newfrag}$,
      Zlength $\mathit{oldfrag}$ < 64 ->
      Zlength $\mathit{newfrag}$ < 64 ->
     (16 | Zlength $\mathit{hashed}$) ->
     (16 | Zlength $\mathit{blocks}$) -> 
     $\mathit{oldfrag}$++$\mathit{msg}$ = intlist_to_Zlist $\mathit{blocks}$ ++ $\mathit{newfrag}$ ->
   update_abs $\mathit{msg}$ (S256abs $\mathit{hashed}$ $\mathit{oldfrag}$) (S256abs ($\mathit{hashed}$++$\mathit{blocks}$) $\mathit{newfrag}$)).
\end{lstlisting}
where \lstinline{intlist_to_Zlist} unpacks big-endian 32-bit words
into a sequence of byte values.

With these preliminaries defined, I can now present the
separation-logic specification of the \lstinline{Update} function.
\begin{lstlisting}
Definition SHA256_Update_spec :=
DECLARE _SHA256_Update
 WITH $a$: s256abs, $\mathit{data}$: list Z, $c$ : val, $d$: val, $\mathit{sh}$: share, $\mathit{len}$ : nat
 PRE [ _c OF tptr t_struct_SHA256state_st, _data_ OF tptr tvoid, _len OF tuint ]
   PROP ($\mathit{len}$ <= length $\mathit{data}$; 
         s256a_len $a$ + Z.of_nat $\mathit{len}$ * 8 < two_p 64)
   LOCAL (`(eq $c$) (eval_id _c); `(eq $d$) (eval_id _data_); 
          `(eq (Z.of_nat $\mathit{len}$)) (`Int.unsigned(`force_int(eval_id _len))))
   SEP(`K_vector (eval_var _K256 (tarray tuint 64));
       `(sha256state_ $a$ $c$); `(data_block $\mathit{sh}$ $\mathit{data}$ $d$))
 POST [ tvoid ] 
  EX $a'$:s256abs, 
   PROP (update_abs (firstn $\mathit{len}$ $\mathit{data}$) $a$ $a'$) LOCAL ()
   SEP(`K_vector (eval_var _K256 (tarray tuint 64));
       `(sha256state_ $a'$ $c$); `(data_block $\mathit{sh}$ $\mathit{data}$ $d$)).
\end{lstlisting}
\lstinline{DECLARE} gives the name (C language function identifier)
of the function being specified.
\lstinline{WITH} binds (logical/mathematical) variables
that can be used in both the precondition and postcondition.

The precondition has the form 
\lstinline{PRE [$\vec{x}$]} \lstinline{PROP $P$} \lstinline{LOCAL $Q$} \lstinline{SEP $R$}
where $\vec{x}$ are the function parameters, annotated
with their C language types (e.g., \lstinline{data_}
has type pointer-to-void); 
$P$ are pure logical \lstinline{PROP}ositions (that do not refer to the input state);
$Q$ are local facts (that do not refer to the memory,
but may refer to program variables), and
$R$ are spatial facts (that refer to the memory
via \lstinline{SEP}aration logic).

Here, $\mathit{data}$ is a sequence of integers
and $d$ is an address in memory; both of these
are logical variables.  The \lstinline{LOCAL} clause
says that the function's \lstinline{data_} parameter
actually contains the value $d$,
the \lstinline{c} parameter contains the
pointer value $c$, and so on.
The \lstinline{SEP} clause names three 
\emph{separate} memory regions of interest:
the 64-word global array $K256$;
a \lstinline{SHA256_CTX} $c$ representing abstract state $a$;
and a data-block at address $d$ containing the
next message-segment $\mathit{data}$.

The postcondition is parameterized by the return
value (this particular function has no return value).
Its \lstinline{PROP} part relates $a$ to the new abstract state $a'$;
the \lstinline{LOCAL} part is empty (since there's no return value
to characterize), and the spatial (\lstinline{SEP}) part says:
the global array \lstinline{K256} is still there unchanged;
at address $c$ there is now an updated state $a'$;
and the data-block at $d$ is still there unchanged.

\paragraph{Corollary}
By the nature of separation logic, this functional
specification inherently makes specific guarantees
about confidentiality, integrity, and 
lack of buffer overruns:
\begin{itemize}
\item The only variables or data structures
read or written to are those 
mentioned in function's specification.
\item The only variables or data structures
written to are those 
mentioned with write permission
in the function's specification.
\item The values written are limited to what
the specification claims.
\end{itemize}
For example, \lstinline{SHA256_Update_spec}'s
\lstinline{SEP} clauses mention only
the memory blocks at addresses 
\lstinline{_K256}, $\mathit{c}$, and $d$;
and the permission-share $\mathit{sh}$ (controlling $d$) 
is not mentioned as writable; this severely limits
where \lstinline{SHA256_Update} can read and write.

Static analysis tools such as Coverity cannot prove functional
correctness, but at least in principle they can find basic safety
problems.  But ``Coverity does not spot the heartbleed flaw ... it
remained stubborn even when they tweaked various analysis
settings. Basically, the control flow and data flow between the socket
read() from which the bad data originates and the eventual bad
memcpy() is just too complicated.'' \cite{regehr14:coverity}

SHA-256, especially the Update function which copies fragments of arrays,
contains nontrivial control flow and data flow leading to memcpy().
But the full verification reported in this paper
\emph{must} fully analyze the control and data, no matter how complicated;
and the higher-order logic (CiC) provides a sufficiently expressive
tool in which to do it.  We know there's no heartbleed in SHA.

\section{Is the specification right?}
\label{sec:specright}
The SHA-256 program is about 235 lines
of C code
(including blank lines and sparse comments).

The FIPS 180-4 specification of SHA is 35 pages of 
text and mathematics; the parts specific to SHA-256
are perhaps 16 pages.  My functional specification---the
``translation'' of this to logic---is 169 lines of Coq,
but it relies on libraries for the mathematical
theories of the unbounded integers, the 32-bit integers,
and lists, which together are many lines more.

My specification of how the C code corresponds
function-for-function to the functional spec
takes 247 lines of Coq.  The proof in Coq is much
larger---see \S\ref{sec:proof}---but it need not be trusted
because it is machine-checked.

Have we gained anything when the specification of a 
235-line program is 169+247 lines?  Is it easier to understand
(and trust) the program, or its specification?

There are several reasons that the specification is valuable:
\begin{enumerate}
\item It can be manipulated logically.  For example, two different
characterizations of the $W$ function can be proved equivalent.
\item The functional specification can be executed on test data.
In this case, we do this indirectly but assuredly by proving its
equivalence to a functional program that executes on the test data.
\item The proof of SHA-256 correctness connects directly to the
proof of C-compiler correctness, \emph{inside the theorem prover
with no specification ``gaps.''}
\end{enumerate}

The last point is quite important.  The C program for SHA-256 may be
only 235 lines, but its meaning depends on the understanding of
the semantics of C.  Even if it is true that (these days) C has a clear and
well-understood specification,\footnote{and even if it were
true that compiler experts understand that specification
\emph{in the same way} that programmers do, which is doubtful \cite{wang2013towards}}
that spec is orders of magnitude larger
than 235 lines.  In contrast, in the Verified Software Toolchain
the C spec is an \emph{internal interface} between the program logic
and the CompCert compiler.  That is, suppose for the sake of argument
that CompCert's C specification is ``wrong;'' or the Verifiable C
program logic is ``wrong.''   It still won't matter:
by composing the correctness proof of SHA-256 (in the Verifiable C program
logic), with the soundness proof of the program logic,
with the correctness proof of CompCert, we get an end-to-end proof
about the observable input/output behavior of the assembly language program,
regardless of the internal specifications.

Still, the first point is important too:  the value of a specification
is that one can interact with it in logic.  You need not assume that
my translation of the SHS document into Coq is correct;
you have the opportunity to test its properties mathematically
(by proving theorems about it) in the proof assistant.

The specifications of the 
\lstinline{block_data_order},
\lstinline{Init}, 
\lstinline{Update}, 
and \lstinline{Final} functions are rather complex,
because of the way they support incremental
hashing.  
One mathematical way of gaining confidence in 
these specifications is to compose them.
That is, this C function
\begin{verbatim}
void SHA256(const unsigned char *d, size_t n, unsigned char *md) {
    SHA256_CTX c;
    SHA256_Init(&c);
    SHA256_Update(&c,d,n);
    SHA256_Final(md,&c);
}
\end{verbatim}
should be equivalent to nonincremental
SHA-256 hashing.  We can see this in its spec:
\begin{lstlisting}
Definition SHA256_spec :=
 DECLARE _SHA256
 WITH $d$: val, $\mathit{len}$: Z, $\mathit{dsh}$: share, $\mathit{msh}$: share, $\mathit{data}$: list Z, $\mathit{md}$: val
 PRE [ _d OF tptr tuchar, _n OF tuint, _md OF tptr tuchar ]
   PROP (writable_share msh; $\quad$ Z.of_nat (length data) * 8 < two_p 64) 
   LOCAL (`(eq $d$) (eval_id _d); `(eq (Z.of_nat (length $\mathit{data}$))) (`Int.unsigned (`force_int (eval_id _n)));
         `(eq $\mathit{md}$) (eval_id _md))
   SEP(`K_vector (eval_var _K256 (tarray tuint 64));
       `(data_block $\mathit{dsh}$ $\mathit{data}$ $d$); `(memory_block msh (Int.repr 32) $\mathit{md}$))
  POST [ tvoid ] 
   SEP(`K_vector (eval_var _K256 (tarray tuint 64));
        `(data_block dsh data $d$); `(data_block msh (SHA_256 data) $\mathit{md}$)).
\end{lstlisting}
This says that calling \lstinline{SHA256($d$,$n$,$\mathit{md}$)}
will fill in the message-digest $\mathit{md}$
with the hash as computed by the functional specification
\lstinline{(SHA_256 $\mathit{data}$)},
as long as the memory at $d$
was indeed $\mathit{data}$,
and $\mathit{data}$ is less than two billion gigabytes.
In addition, the (global) \lstinline{K256} must be 
properly initialized beforehand, and is guaranteed
preserved unchanged; the input data must be present
and will not be modified;
and the output area must be writable.
Finally, no other memory (except for the activation
records of called functions) will be read or written;
this is an implicit (but very real) guarantee
of the separation logic.

The fact that \lstinline{SHA256} satisfies this
(relatively) simple specification is a \emph{proof}
for the nonincremental case,
that all the other functions' specifications
are ``right''---that is, they compose properly.
It is not a proof that the incremental case
(more than one call to \lstinline{Update})
is specified right, but it does help build assurance.
Guarantees about the incremental case
rely on the rightness of the
\lstinline{SHA256_Update_spec} definition.

\section{The proof}
\label{sec:proof}
The proof of the functional program w.r.t. the functional
specification is fairly concise:

\noindent \begin{tabular}{rr p{4in}}
Lines & Seconds & component \\ \hline
~\\[-2ex]
1022 & 29 & Lemmas about the functional spec \\
1202 & 12 & Correctness proof of functional program \\ \hline
~\\[-2ex]
2424 & 41 & \emph{Total}\\ [2ex]
\end{tabular}

\noindent
I show here the size of the Coq proof, and the
time for Coq to check the proof in batch mode.
The first component (lemmas) is shared with
the proof of the C program.

My correctness proof of the C program
is quite large---6539 lines of proof, written by hand
with some cut-and-paste.
It's also very slow to check.  Thus, the current
Verifiable C system must be regarded as a prototype implementation
(though unlike most prototypes it has a machine-checked proof
of correctness).

\noindent \begin{tabular}{rr p{4in}}
~\\[-1ex]
Lines & Seconds & component \\ \hline
~\\[-2ex]
1022 & 29 & Lemmas about the functional spec \\
229 & 83 & Proof of addlength function \\
1640 & 625 & sha256\_block\_data\_order()\\
43 & 256 & SHA256\_Init() \\
1682 & 800 & SHA256\_Update() \\
1484 & 687 & SHA256\_Final() \\
58 & 91 & SHA256() \\ \hline
~\\[-2ex]
6539 & 2571 & \emph{Total}\\ [2ex]
\end{tabular}

Writing 6500 lines to verify this program is simply too much work.
I'm sure that my proof is clumsy and inelegant in many places
and likely there is a 2000-line proof struggling to get out.
But perhaps the real solution here is to drastically improve the
proof automation by using modern proof-search algorithms
such as satisfiability modulo theories (SMT).
Recent experiments have combined the trustworthiness of Coq
(small kernel checking a proof)
with the power of SMT (large C++ program claiming unsatisfiability)
by exporting proof witnesses from SMT solvers to Coq
\cite{besson11,armand11}.

Checking the proofs takes 2571 seconds (43 minutes)
on one processor of an Intel core i7 (at 3.4 Ghz) with plenty of
cache and 2GB ram.  Multicore, it goes much faster
using parallel make.  Still, 43 minutes is far too long
for a program this size.  As a batch command it might
be tolerable; the problem is in the interactive proof,
where it might take 2 minutes to move past one
line of C code---this is not very interactive.

Coq is not normally so slow; the problem is
symbolic execution.  The component listed in the
first line of the table above (lemmas about the
functional spec etc.) contains no
symbolic execution---no application of the
C-language program logic with all its side conditions.
That component takes only 29 seconds in Coq.

Why is symbolic execution of C programs so slow?
We wrote the prototype interactive prover for Verifiable C
as a user-driven symbolic executor programmed in the
Ltac language of Coq.  As symbolic execution proceeds,
the user frequently provides proofs for those steps
where the automation cannot find a proof.  
So far, no problem---although it would be better
to have more automation, so less user interaction; that's
future work.  The problem is that Coq builds
a proof trace of the symbolic execution---that is,
every step of the analysis corresponds to a data structure
describing a proof term.  Worse yet, as the terms
are being constructed they are actually function
closures (activation records) that (when invoked)
will build the concrete proof terms.
Since each step of symbolic execution
checks dozens of conditions, the proof-construction
function closures can occupy hundreds of megabytes.
Since I am running a 32-bit Coq limited to 2 gigabytes
and with a copying garbage collector, this consumes
most of memory.

One solution, readily available, is to run 64-bit Coq
on my 32-gigabyte desktop computer.  But I would
like to think that verifying a program as small
as SHA-256 could readily be done on an ordinary laptop.
Two other solutions to this problem are potentially
available:
\begin{enumerate}
\item Program the symbolic execution using computational
reflection.  That is, write a functional program in Coq
to do symbolic execution of the Verifiable C program logic,
prove it correct in Coq, and then apply it to C programs
such as SHA-256.  During the execution of such a program,
Coq does not build proof traces.  
The VST project is already
working on this approach \cite[Chapters 25,46,47]{appel14:plcc}.
\item Coq does not necessarily need to build proof terms
as data structures.  The Edinburgh LCF proof assistant
in 1979 demonstrated a technique for using a
small trustworthy proof-checking kernel using an
abstract-data-type interface rather than
a proof-term data structures \cite{gordon79}.
Some modern systems such as HOL and Isabelle/HOL also use this
ADT approach.
Using ADTs instead of proof terms
would solve the memory problem,
but it would require substantial change inside Coq.
\end{enumerate}

\section{Is it really OpenSSL?}
\label{sec:changes}
Verifiable C is a subset of the C language;
certain program transformations are needed before
applying the program logic.  Therefore I modified the
OpenSSL implementation of SHA-256 in these ways:

\begin{enumerate}
\item OpenSSL is heavily macro-ized so that the same
source file generates SHA-224, SHA-256, and other
instantiations.  I expanded the macros and included header
files just enough to specialize to SHA-256.
\item Verifiable C prohibits side effects inside subexpressions;
I broke these into separate statements.
\item Verifiable C
prohibits memory references inside subexpressions,
and
requires that each assignment statement have at most one memory reference
at top level.  This requires \emph{local} rewriting, often with
the introduction of a temporary variable.
\item The current prototype Verifiable C requires a return statement instead
of a fall-through at the end of a function, so I added some return statements.
\end{enumerate}
The first two of these are handled automatically by the compiler
\emph{before} applying the program logic, so they do not require
any manual changes to the program.  The third one could
also be handled by the compiler but is not at present.
The last one will be remedied in the near future.

Still, by instantiating some macros I made my proof task easier,
at the cost of limiting the generality of my result to the 256-bit case.

\section{Composing this proof with others}
\label{context}

Verifications of individual components can suffer from the
problem that the size of the specification can be larger than
the size of the program.  The machine-checked proof removes
the program from the trusted base, but if the specification
is as big as the program, what have we gained?
The answer can come in the composition of systems.
When we compose the SHA-256 proof with the Verifiable C
proof with the CompCert proof, the entire specification
of C drops out of the trusted base (as explained in \S\ref{sec:specright}).

At the other end, we should connect SHA-256 with its application,
for example in the HMAC protocol for cryptographic authentication.
HMAC calls upon a cryptographic hash function (such as SHA-256).

Desired claim:  A particular implementation $A$ of HMAC in the C language
is a key-selected pseudorandom function (PRF) on the message.

Proof structure:
\begin{enumerate}
\item The C program $A$, which calls upon SHA-256,
correctly implements the functional specification of HMAC.
(Future work: To be formalized and proved using techniques similar to 
those described in this paper.)
\item The functional spec of HMAC (indexed by a randomly chosen key)
gives a PRF, provided that
the underlying hash primitive is a Merkle-Damg\r{a}rd hash construction
applied to a PRF compression function.  Proof: 
Future work based
(for example) on 
\citeN{gazi2014exact}
or
Bellare \emph{et al.} 
\citeyear{bellare1996keying,bellare2006new}
but fully formalized in Coq.\footnote{%
There are rumors that something like this has been done in
CertiCrypt, but no publication describes a CertiCrypt
proof of HMAC or NMAC.  There is a brief description of
an EasyCrypt proof of NMAC in \cite{barthe2012:itp},
but (unlike CertiCrypt) EasyCrypt is not foundational:
``EasyCrypt was conceived as a front-end
to the CertiCrypt framework. \ldots Certification remains
an important objective, although the proof-producing mechanism
may fall temporarily out of sync with the development of EasyCrypt.''
\cite{barthe2012:itp}  It appears that EasyCrypt/CertiCrypt have been
continuously out of sync since 2012.}
\item \textbf{OpenSSL's SHA-256 correctly implements the functional
spec of SHA-256.}  (This paper.)
\item The functional spec of SHA-256 is a 
Merkle-Damg\r{a}rd hash construction.
(Provable from the functional spec described in this paper; future work.)
\item The compression function underlying SHA-256 is a PRF.\footnote{
More precisely:
Bellare's \citeyear{bellare2006new} HMAC proof 
requires that the underlying hash function $H$
is a Merkle-Damg\r{a}rd construction on a round function
$R(x,m)$ such that:  (A) the ``dual family'' of $R$ is secure against
related-key attacks, and (B) $R$ is a pseudorandom function (PRF).
That is, given a random unknown key $x$, it is computationally 
intractable to distinguish $\lambda m.R(x,m)$ from a randomly chosen
function over $m$.}
Oops!  Nobody knows how to prove that 
SHA-256's compression function is a PRF.  For now, the world survives
on the fact that nobody knows how (without knowing the key)
to distinguish it from a random function.
\end{enumerate}
So this is a chain of proofs with a hole.  Fortunately,
the hole is in \emph{only} the place where symmetric crypto
always has a hole:  the crypto properties of the
symmetric-key primitive.  This hole is bounded very closely
by the \emph{functional specifications} of both
SHA-256 (169 lines of Coq) and of one-way functions.
All the rest---the messy parts in C---are not in the trusted
base, their connection is \emph{proved} with machine-checked proofs.

That's \emph{almost} true---but there's really one more thing.
Eventually the end user has to call
the HMAC function, from a C program.  That user will need a specification
related to C-language calling conventions.  In this paper I
have shown what such a specification looks like for SHA-256:
the definition \lstinline{SHA256_spec}
in \S\ref{sec:specright}.  This is not too large or complex;
the specification of HMAC would be similar.

\section{The trusted base}
The assurance of the correctness of SHA-256 relies on a \emph{trusted base,}
a series of specifications and implementations that must be correct
for the proof to be meaningful.
That is, we must trust:

\begin{description}
\item[\hspace*{-1em}Calculus of Inductive Constructions]
The logic underlying Coq must be consistent in order to trust proofs in it.
Several refereed papers have been published giving consistency arguments
for versions of this logic,
but these papers have not fully tracked the particular logic implemented
in Coq.  In general, CiC is a stronger and more complex logic than
some of its competitors such as HOL and LF, so there is more to trust here.

\item[\hspace*{-1em}Axioms]
The soundess proof of Verifiable C uses the axioms of
(dependent) functional extensionality
and propositional extensionality.
Both of these axioms are consistent
with Coq's core theory, that is,
when they are added to CiC it is still impossible to prove \emph{false.}
But in 2013 it was discovered that propositional extensionality
is unsound in Coq 8.4.\footnote{Daniel Schepler, 
Maxime D\'en\`es, Arthur Chargu\'eraud,
``Propositional extensionality is inconsistent in Coq,''
coq-club mailing list, 12 December 2013.}
This is not an inherent problem with the consistency of
the axiom, it is a bug in Coq's termination checking.\footnote{
``Just to reassure everyone having developments relying on these
  axioms, the problem does not seem too deep. There is a slight
  inconsistency in the way the guard checker handles unreachability
  hypotheses, but that should be easily fixed without too much impact
  on existing contributions (hopefully).''
Maxime D\'en\`es, coq-club mailing list, 12 December 2013.}
It is expected that near-future releases of Coq will be consistent with
functional and propositional extensionality.

\item[\hspace*{-1em}Coq kernel]
Coq has a \emph{kernel} that implements proof-checking (type-checking)
for CiC.  We must trust that this kernel is free of bugs.
The kernel is between 10,000 and 11,000 lines of ML code.

\item[\hspace*{-1em}OCaml compiler]
The Coq kernel's ML compiler is compiled by a the OCaml
compiler, which is tens of thousands of lines of code.
That compiler is compiled by itself, leading to an
infinite regression that cannot be fully trusted \cite{thompson84}.

\item[\hspace*{-1em}OCaml runtime]
Coq, running as an OCaml-compiled binary, is serviced
by the OCaml runtime system and garbage collector,
written in C.

\item[\hspace*{-1em}Functional specification of SHA-256]
We must trust that I have correctly transcribed the
FIPS 180-4 standard into Coq.  However, if (in the future)
we complete crypto proofs \emph{about} the functional spec,
then this element drops out of the trusted base, to some extent.

\item[\hspace*{-1em}API specification of SHA-256]
The intended relation of the functional spec to data structures
at API function calls, what I have called the ``API spec'',
must be right, otherwise I have proved the wrong thing.

\item[\hspace*{-1em}Specification of CompCert]
I have explained earlier that we do not need to trust
CompCert's specification of the C language, since that 
``drops out'' of the trusted base when composed with
the soundness proof of the Verifiable C program logic.
But we do need to trust that  CompCert's specification
of Intel x86 (IA-32) assembly language is correct.

\item[\hspace*{-1em}Assembler]
At present, there is no proved-correct assembler for CompCert.
The transition from assembly language to machine language is done
by the GNU assembler and linker.  Proving correctness
of an assembler is quite achievable \cite{wu03:flit}.

\item[\hspace*{-1em}Intel Core i7]
The OCaml binary and garbage collector run in machine
language.  My computer is an Intel Core i7, and we must
trust that Intel has correctly implemented the instruction-set
architecture, for two reasons:  First, we run Coq on it,
so that affects confidence in the proof-checking.
Second, we run the SHA-256 on it, and therefore the specification
of the assembly language is part of the assumptions 
of the CompCert correctness proof.

\end{description}

This is a long chain of trust.  One can do \emph{much}
better.  The Foundational Proof-Carrying code project
had a trusted base of less than 3000 lines of code,
including axioms, proof-checking kernel, compiler, runtime,
functional specification, API specification, and ISA
specification \cite{wu03:flit}
(and it avoided the Thompson paradox by not needing a compiler
in the trusted base).
But that was for a much less ambitious project (safety
instead of correctness) in a much weaker logic (LF
instead of Coq).  Scaling those tiny-trusted-base techniques
to VST would not be easy.

\section{Related work in Crypto}

One can compare to previous work on several dimensions:
\begin{description}
\item[Specification]  Is there a specification
of the program's function?
Is the specification be written in a
pure functional (or relational) language, amenable to analysis in a proof assistant?
(That is, \emph{aside} from verification that an implementation satisfies
the functional spec, can one reason about the functional spec \emph{per se}?)
\item[Implementation]  Is the proof about an efficient implementation
(e.g., in compiled C or Java), or only about a functional spec?
\item[Foundational]  Is there an end-to-end machine-checked proof 
from the foundations of logic, that the 
generated assembly code correctly implements the specification,
without trusting the compiler (or equivalently, without trusting
the programming-language specification)?  (Where there is no 
implementation, this question does not even apply.)
\item[Automatic]  Does the verifier check or synthesize the crypto algorithm
without much (or any) interactive (or scripted) human input or annotations?
\item[General] Can the verifier handle all parts of a crypto algorithm
(such as the management code in \lstinline{SHA256_Update}), or only
the parts where the number of input bits is fixed
and the loops can be completely unrolled?
\end{description}

\paragraph{Specification, implementation, foundational, not \st{automatic}, general:}
The work described in this paper.

\paragraph{Specification, implementation, not \st{foundational}, automatic, not \st{general}:}
\citeN{smith2008automatic} verified several block-cipher implementations
written in Java, w.r.t. a functional spec written either in Java or in ACL2.
They compiled to byte-code, then 
used a subset model of the JVM to generate straight-line code.
This renders them immune to bugs in \textsf{javac},
but the JIT compiler (from byte-code to native code) is unverified
(or, equivalently, their JVM spec is unverified).
They prove the straight-line code equivalent to the straight-line code
of the functional spec.
Their verification is fully automatic, using rewrite rules to simplify
and normalize arithmetic expressions---with rules for many special patterns that
occur in crypto code, such as bitfield concatentation by shift-and-or.
After rewriting, they use a SAT solver to compare the normalized expressions.
Smith and Dill's method applies only where the number of input bits is fixed
and the loops can be completely unrolled.  Their verifier would likely be applicable
to the SHA-256 block shuffle (\lstinline{sha256_block_data_order}) function, but certainly not
to the management code (\lstinline{SHA256_Update}).

\paragraph{Limited specif\st{ication}, implementation,  not \st{foundational}, automatic, not \st{general}:}
Cryptol \cite{erkok2009hardware} generates C or
VHDL directly from a functional specification,
where the number of input bits is fixed
and the loops can be completely unrolled. 
In fact, Cryptol \emph{does} unroll the loop in 
\lstinline{sha256_block_data_order},
leading to a program that is 1.5x faster than OpenSSL's standard
implementation (when both are compiled by gcc and run on Intel Core i7)\footnote{
Aaron Tomb, Galois.com, personal communication, 13 January 2014.}.
The spec is in a Haskell dialect, not directly embeddable in any
existing proof assistant; the synthesizer is not verified.

\paragraph{Specification, implementation, not \st{foundational}, not \st{automatic}, not \st{general}:}
\citeN{toma2005formal}
used ACL2 to prove correctness of a VHDL implementation of the SHA-1
block-shuffle algorithm.

\paragraph{No \st{specification}, implementation, not \st{foundational}, automatic, general:}
One can apply static analysis algorithms to C programs to learn whether
they have memory-safety bugs such as buffer overruns.  Many such analyses
are both unsound (will miss some bugs) and incomplete (will report
false-positives about bug-free programs).  Even so, they can be very useful;
but they do not attempt to prove functional correctness with respect
to a specification.

\paragraph{Complementary work}
In this paper I have concentrated on the verification that an
implementation satisfies its functional specification.  
Complementary work establishes properties of the functional specs.
For example, \citeN{duan2005functional}
proved a property of the functional specs of several encryption algorithms:
that decryption is the inverse of encryption.
More relevant to SHA-256, \citeN{backes2012verified} verify mechanically
(in EasyCrypt) that Merkle-Damg{\aa}rd constructions have certain 
security properties.
Bellare \citeyear{bellare1996keying,bellare2006new}
gave the first proofs of NMAC\-/\-HMAC security 
 (without a machine-checked proof);
\citeN{gazi2014exact}
prove PRF-security of NMAC\-/\-HMAC (without a machine-checked proof),
based on fewer assumptions.

\paragraph{EasyCrypt}
\citeN{almeida2013certified} describe the use of their EasyCrypt tool
to verify the security of an implementation of the RSA-OAEP encryption scheme.
A functional specification of RSA-OAEP is written in EasyCrypt, which then verifies its
security properties.  An unverified Python script translates the EasyCrypt 
specification to (an extension of) C; then 
an extension of CompCert compiles it to assembly language.  Finally,
a leakage tool verifies that the assembly-language program has no more
program-counter leakage than the source code, i.e. that the compiled
program's trace of conditional branches is no more informative to the adversary
than the source code's.

The EasyCrypt verifier is not fully foundational; it is an OCaml program
whose correctness is not proved.  The translation from EasyCrypt to C is
not foundational.  The translation
from C to assembly language is foundational, using CompCert.
Programs must operate on fixed-size data blocks.

The leakage model is the Program Counter Model (trace of conditional branches),
and there is a foundational checker (i.e., proved correct  in Coq) that
compiled programs leak no more PC-trace information than the source program.
But other forms of leakage are not modeled.  In particular, SHA-256
has a line of code (marked \verb|/* keep it zeroed */|) whose entire purpose 
is to reduce leakage of message fragments through deallocated memory;
but that kind of leakage channel is not modeled in EasyCrypt.

EasyCrypt's C code relies on bignum library functions called through
a nonstandard specification interface---nonstandard, because
standard CompCert (through the current version, 2.3) has no way to 
handle external function calls that receive or return results in memory.
But EasyCrypt provides no mechanism by which
these functions can be proved correct, nor does it give a 
mechanized proof theory for this custom specification interface.

In summary, Almeida \emph{et al.} attack two problems that are exactly complementary
to the work I have done.  EasyCrypt allows reasoning about crypto properties of
the functional spec; 
and they extend CompCert's compilation-correctness guarantees
with a guarantee about a particular side channel.
They do not reason about the semantic relation between the functional
spec and the C program (either the main algorithm or the bignum library).

\section{Related work in C verification}
There are many program analysis tools for C.  Most of them do not
address functional specification or functional correctness, and most
are unsound and incomplete.  Nonetheless, they are very useful in practice:
C static-analysis tools are a billion-dollar-a-year industry.\footnote{Andy
  Chou, ``From the Trenches: Static Analysis in Industry'', invited
  talk at POPL 2014, January 24, 2014.}

Foundational formal verification of C programs has only
recently been possible.  The most significant such works are
both operating-system kernels:  seL4
\cite{klein2009sel4} and 
CertiKOS \cite{Gu:2011:CCK:2103799.2103803}.
Both proofs are refinement proofs between functional specifications
and operational semantics.  Both proofs are done in
higher-order logics:  seL4 in Isabelle/HOL and CertiKOS in
Coq.  Each of these projects verifies a significantly larger
C program than the SHA-256 program I describe here.

Neither of their proof frameworks use separation logic,
neither can accommodate the use of addressable local variables
in C, and neither can handle function-pointers or higher-order
specifications.  This means that the OpenSSL SHA-256 program
could not be proved in these frameworks, because it uses
addressable local variables.  A minor
adjustment of the C program---moving the X array into 
the \lstinline{SHA256state_st} structure---would
eliminate the use of addressable locals, however.

SHA-256 does not use function pointers.
However, OpenSSL uses function pointers
in its ``engines'' mechanism, an object-oriented style
of programming that dynamically connects components together---for 
example, HMAC and SHA.  The \emph{Verifiable C} program logic,
with higher-order separation logic, is capable of reasoning
about such object-oriented patterns in C \cite[Chapter 29]{appel14:plcc}.

The C semantics used in the original seL4 proof was not connected to a
C compiler (e.g., it is not the CompCert C semantics), so the entire C
semantics is part of the trusted base of the seL4 proof.  More recent
work removes C from the trusted base:
\citeN{sewell2013translation}
perform translation validation for \texttt{gcc 4.5.1}
by decompiling ARM code to logical graphs, then using a combination of
carefully tuned heuristic proof search and SMT solving to find a proof
of equivalence between source program and machine language.

CertiKOS is proved correct with respect to the CompCert C semantics,
so (as in my SHA-256 proof) this C semantics drops out of the trusted base.

Both seL4 and CertiKOS are newly written C programs designed for
verification.  In principle, this is the right way to do things:
it can be difficult to verify pre-existing programs.
However, there are times when it's important to be
able to do so.  Aircraft manufacturers, who have code bases already
certified (and trusted) for use in passenger jets,
should not be asked to rewrite their fly-by-wire software
just so that they can apply new and better verification techniques.
And, to the extent that the security community has come to trust
nonfunctional properties of OpenSSL---lack of timing channels,
fault injection resistance, compatibility with many C compilers---this
trust cannot necessarily be transferred to new implementations.

\section{Conclusion}

Functional correctness verification of C programs has important
applications in computer security.
Correctness  has the corollary
of memory safety, which is valuable in itself.
But in the implementatation of protection mechanisms (such as 
operating systems, encryption, authentication), safety is not
enough: correctness is what guarantees that these mechanisms
actually secure the systems that they are supposed to protect.

C is not friendly
to program verification:  it has tricky corners,
one needs to keep track of many side conditions.
Nonetheless it is possible to do full formal verification
of C programs.  Previous results have demonstrated this
for operating-system microkernels \cite{klein2009sel4} (though not with
the verified connection to a verified compiler).
In such results, the C program is typically constructed anew
with a design particularly suited for the verification task.

In this project I demonstrated that one can verify a program \emph{as it is}.
This is valuable because widely used open-source cryptographic primitives
have many relevant properties other than functional correctness.
For example, the comment \verb|/* keep it zeroed */|
in the \lstinline{Update} function is attached to a line
that has no functional impact, but might reduce information
flow through deallocated variables.  My program logic
cannot prove that this line reduces side channels,
but at least I can prove that it
does not impair functional correctness.
Whatever assurance and confidence the community has gained
in this program will only be increased by this verification.

At the same time, these cryptographic primitives have many 
implementation variants (using machine-dependent instructions).  
Small variations of this proof
can serve to prove all of them equivalent to the same
functional specification, even if there are not ``many eyes''
on every single one of them to keep the bugs shallow.



\bibliographystyle{acm-Reference-Format-Journals}
\bibliography{appel}

\clearpage
\appendix
\section*{APPENDIX}
\section{SHA-256, C program adapted from OpenSSL}
%\begin[language=C]{lstlisting}
{\small \begin{verbatim}
/* Adapted 2013 from OpenSSL098 crypto/sha/sha256.c     Copyright (c) 2004 The OpenSSL 
 * Project.  All rights reserved according to the OpenSSL license.
 */

extern unsigned int __builtin_read32_reversed(const unsigned int * ptr);
extern void __builtin_write32_reversed(unsigned int * ptr, unsigned int x);

#include <stddef.h>
#include <string.h> /* for memcpy, memset */

#define HOST_c2l(c,l)   \
   (l=(unsigned long)(__builtin_read32_reversed (((unsigned int *)c))),c+=4,l)

#define HOST_l2c(l,c)   \
   (__builtin_write32_reversed (((unsigned int *)(c)),l),c+=4,l)

#define SHA_LONG unsigned int

#define SHA_LBLOCK      16
#define SHA_CBLOCK      (SHA_LBLOCK*4)  
   /* SHA treats input data as a contiguous array of 32 bit wide big-endian values. */
#define SHA_LAST_BLOCK  (SHA_CBLOCK-8)
#define SHA_DIGEST_LENGTH 20

#define SHA256_DIGEST_LENGTH    32

typedef struct SHA256state_st {
        SHA_LONG h[8];
        SHA_LONG Nl,Nh;
        unsigned char data[SHA_CBLOCK];
        unsigned int num;
        } SHA256_CTX;

#define MD32_REG_T long
#define ROTATE(a,n) (((a)<<(n))|(((a)&0xffffffff)>>(32-(n))))

static const SHA_LONG K256[64] = {
        0x428a2f98UL,0x71374491UL, ... ... 0xbef9a3f7UL,0xc67178f2UL };

/* FIPS specification refers to right rotations, while our ROTATE macro is left one. 
 * This is why you might notice that rotation coefficients differ from those 
 * observed in FIPS document by 32-N...
 */
#define Sigma0(x)   (ROTATE((x),30) ^ ROTATE((x),19) ^ ROTATE((x),10))
#define Sigma1(x)   (ROTATE((x),26) ^ ROTATE((x),21) ^ ROTATE((x),7))
#define sigma0(x)   (ROTATE((x),25) ^ ROTATE((x),14) ^ ((x)>>3))
#define sigma1(x)   (ROTATE((x),15) ^ ROTATE((x),13) ^ ((x)>>10))

#define Ch(x,y,z)   (((x) & (y)) ^ ((~(x)) & (z)))
#define Maj(x,y,z)  (((x) & (y)) ^ ((x) & (z)) ^ ((y) & (z)))

void sha256_block_data_order (SHA256_CTX *ctx, const void *in) {
  unsigned MD32_REG_T a,b,c,d,e,f,g,h, s0,s1,T1,T2,t;
  SHA_LONG      X[16],l,Ki;
  int i;
  const unsigned char *data=in;

  a = ctx->h[0];  b = ctx->h[1];  c = ctx->h[2];  d = ctx->h[3];
  e = ctx->h[4];  f = ctx->h[5];  g = ctx->h[6];  h = ctx->h[7];

  for (i=0;i<16;i++) {
      HOST_c2l(data,l); X[i] = l;
      Ki=K256[i];
      T1 = l + h + Sigma1(e) + Ch(e,f,g) + Ki;
      T2 = Sigma0(a) + Maj(a,b,c);
      h = g;    g = f;    f = e;    e = d + T1;
      d = c;    c = b;    b = a;    a = T1 + T2;
  }

  for (;i<64;i++) {
      s0 = X[(i+1)&0x0f];    s0 = sigma0(s0);
      s1 = X[(i+14)&0x0f];   s1 = sigma1(s1);
      T1 = X[i&0xf];
      t = X[(i+9)&0xf];
      T1 += s0 + s1 + t;
      X[i&0xf] = T1;
      Ki=K256[i];
      T1 += h + Sigma1(e) + Ch(e,f,g) + Ki;
      T2 = Sigma0(a) + Maj(a,b,c);
      h = g;    g = f;    f = e;    e = d + T1;
      d = c;    c = b;    b = a;    a = T1 + T2;
  }

  t=ctx->h[0]; ctx->h[0]=t+a;
  t=ctx->h[1]; ctx->h[1]=t+b;
  t=ctx->h[2]; ctx->h[2]=t+c;
  t=ctx->h[3]; ctx->h[3]=t+d;
  t=ctx->h[4]; ctx->h[4]=t+e;
  t=ctx->h[5]; ctx->h[5]=t+f;
  t=ctx->h[6]; ctx->h[6]=t+g;
  t=ctx->h[7]; ctx->h[7]=t+h;
  return;
}

void SHA256_Init (SHA256_CTX *c) {
  c->h[0]=0x6a09e667UL; c->h[1]=0xbb67ae85UL;
  c->h[2]=0x3c6ef372UL; c->h[3]=0xa54ff53aUL;
  c->h[4]=0x510e527fUL; c->h[5]=0x9b05688cUL;
  c->h[6]=0x1f83d9abUL; c->h[7]=0x5be0cd19UL;
  c->Nl=0;    c->Nh=0;
  c->num=0;
  return;
}

void SHA256_addlength(SHA256_CTX *c, size_t len) {
  SHA_LONG l, cNl,cNh;
  cNl=c->Nl; cNh=c->Nh; 
  l=(cNl+(((SHA_LONG)len)<<3))&0xffffffffUL;
  if (l < cNl) /* overflow */
      {cNh ++;}
  cNh += (len>>29);
  c->Nl=l; c->Nh=cNh;
  return;
}

void SHA256_Update (SHA256_CTX *c, 
                const void *data_, size_t len) {
  const unsigned char *data=data_;
  unsigned char *p;
  size_t   n, fragment;

  SHA256_addlength(c, len);
  n = c->num;
  p=c->data;
  if (n != 0)    {
    fragment = SHA_CBLOCK-n;
    if (len >= fragment)  {
       memcpy (p+n,data,fragment);
       sha256_block_data_order (c,p);
       data  += fragment;
       len   -= fragment;
       memset (p,0,SHA_CBLOCK); /* keep it zeroed */
    }
    else  {
       memcpy (p+n,data,len);
       c->num = n+(unsigned int)len;
       return;
    }
  }
  while (len >= SHA_CBLOCK) {
      sha256_block_data_order (c,data);
      data += SHA_CBLOCK;
      len  -= SHA_CBLOCK;
  }
  c->num=len;
  if (len != 0) {
      memcpy (p,data,len);
  }
  return;
}

void SHA256_Final (unsigned char *md, SHA256_CTX *c)  {
  unsigned char *p = c->data;
  size_t n = c->num;
  SHA_LONG cNl,cNh;

  p[n] = 0x80; /* there is always room for one */
  n++;

  if (n > (SHA_CBLOCK-8)) {
        memset (p+n,0,SHA_CBLOCK-n);
        n=0;
        sha256_block_data_order (c,p);
        }
  memset (p+n,0,SHA_CBLOCK-8-n);

  p += SHA_CBLOCK-8;
  cNh=c->Nh; (void)HOST_l2c(cNh,p);
  cNl=c->Nl; (void)HOST_l2c(cNl,p);
  p -= SHA_CBLOCK;
  sha256_block_data_order (c,p);
  c->num=0;
  memset (p,0,SHA_CBLOCK);
  {unsigned long ll;
   unsigned int  xn;
   for (xn=0;xn<SHA256_DIGEST_LENGTH/4;xn++)    
        {   ll=(c)->h[xn]; HOST_l2c(ll,md);   }
  }
  return;
}

void SHA256(const unsigned char *d, 
            size_t n, unsigned char *md) {
  SHA256_CTX c;
  SHA256_Init(&c);
  SHA256_Update(&c,d,n);
  SHA256_Final(md,&c);
  return;
}



\end{verbatim}}
%\end{lstlisting}


\section{The specification}
\label{sec:the-spec}

\begin{lstlisting}
Definition big_endian_integer (contents: Z -> int) : int :=
  Int.or (Int.shl (contents 0) (Int.repr 24))
  (Int.or (Int.shl (contents 1) (Int.repr 16))
   (Int.or (Int.shl (contents 2) (Int.repr 8))
      (contents 3))).

Definition LBLOCKz : Z := 16. (* length of a block, in 32-bit ints *)
Definition CBLOCKz : Z := 64. (* length of a block, in characters *)

Definition s256state := (list val * (val * (val * (list val * val))))%type.
Definition s256_h (s: s256state) := fst s.
Definition s256_Nl (s: s256state) := fst (snd s).
Definition s256_Nh (s: s256state) := fst (snd (snd s)).
Definition s256_data (s: s256state) := fst (snd (snd (snd s))).
Definition s256_num (s: s256state) := snd (snd (snd (snd s))).

Inductive s256abs :=  (* SHA-256 abstract state *)
 S256abs: forall (hashed: list int)   (* words hashed, so far *)
           (data: list Z),  (* bytes in the partial block not yet hashed *)
           s256abs.

Definition s256a_regs (a: s256abs) : list int :=
 match a with S256abs hashed data  => hash_blocks init_registers hashed  end.

Definition s256a_len (a: s256abs) : Z := 
  match a with S256abs hashed data => (Zlength hashed * 4 + Zlength data) * 8   end%Z.

Definition hilo hi lo :=  (Int.unsigned hi * Int.modulus + Int.unsigned lo)%Z.
Definition isbyteZ (i: Z) := (0 <= i < 256)%Z.

Definition s256_relate (a: s256abs) (r: s256state) : Prop :=
     match a with S256abs hashed data =>
         s256_h r = map Vint (hash_blocks init_registers hashed) 
       /\ (exists hi, exists lo, s256_Nh r = Vint hi /\ s256_Nl r = Vint lo /\
             (Zlength hashed * 4 + Zlength data)*8 = hilo hi lo)%Z
       /\ s256_data r = map Vint (map Int.repr data)
       /\ (Zlength data < CBLOCKz /\ Forall isbyteZ data)
       /\ (LBLOCKz | Zlength hashed)
       /\ s256_num r = Vint (Int.repr (Zlength data))
     end.

Definition init_s256abs : s256abs := S256abs nil nil.

Definition sha_finish (a: s256abs) : list Z :=
 match a with S256abs hashed data => SHA_256 (intlist_to_Zlist hashed ++ data)  end.

Definition cVint (f: Z -> int) (i: Z) := Vint (f i).

Definition sha256_length (len: Z)  (c: val) : mpred :=
   EX lo:int, EX hi:int, 
     !! (hilo hi lo = len) &&
     (field_at Tsh t_struct_SHA256state_st _Nl (Vint lo) c *
      field_at Tsh t_struct_SHA256state_st _Nh (Vint hi) c).

Definition sha256state_ (a: s256abs) (c: val) : mpred :=
   EX r:s256state,   !!  s256_relate a r  &&  data_at Tsh t_struct_SHA256state_st r c.

Definition tuints (vl: list int) := ZnthV tuint (map Vint vl).
Definition tuchars (vl: list int) :=  ZnthV tuchar (map Vint vl).

Definition data_block (sh: share) (contents: list Z) :=
  !! Forall isbyteZ contents &&
  array_at tuchar sh (tuchars (map Int.repr contents))
           0 (Zlength contents).

Definition __builtin_read32_reversed_spec :=
 DECLARE ___builtin_read32_reversed
  WITH p: val, sh: share, contents: Z -> int
  PRE [ 1 OF tptr tuint ] 
     PROP() LOCAL (`(eq p) (eval_id 1))
     SEP (`(array_at tuchar sh (cVint contents) 0 4 p))
  POST [ tuint ] 
     local (`(eq (Vint (big_endian_integer contents))) retval) &&
     `(array_at tuchar sh (cVint contents) 0 4 p).

Definition __builtin_write32_reversed_spec :=
 DECLARE ___builtin_write32_reversed
  WITH p: val, sh: share, contents: Z -> int
  PRE [ 1 OF tptr tuint, 2 OF tuint ] 
     PROP(writable_share sh)
     LOCAL (`(eq p) (eval_id 1);
            `(eq (Vint(big_endian_integer contents))) (eval_id 2))
     SEP (`(memory_block sh (Int.repr 4) p))
  POST [ tvoid ] 
     `(array_at tuchar sh (cVint contents) 0 4 p).

Definition memcpy_spec :=  (* elided *)
Definition memset_spec :=  (* elided *)

Definition K_vector : environ -> mpred :=
    array_at tuint Tsh (tuints K) 0 (Zlength K).

Definition sha256_block_data_order_spec :=
  DECLARE _sha256_block_data_order
  WITH hashed: list int, b: list int, ctx : val, data: val, sh: share
  PRE [ _ctx OF tptr t_struct_SHA256state_st, _in OF tptr tvoid ]
    PROP(Zlength b = LBLOCKz; (LBLOCKz | Zlength hashed)) 
    LOCAL (`(eq ctx) (eval_id _ctx); `(eq data) (eval_id _in))
    SEP (`(array_at tuint Tsh  
            (tuints (hash_blocks init_registers hashed)) 0 8 ctx);
       `(data_block sh (intlist_to_Zlist b) data);
       `K_vector (eval_var _K256 (tarray tuint 64)))
  POST [ tvoid ]
     (`(array_at tuint Tsh  
          (tuints (hash_blocks init_registers (hashed++b))) 0 8 ctx) *
      `(data_block sh (intlist_to_Zlist b) data) *
      `K_vector (eval_var _K256 (tarray tuint 64))).
 
Definition SHA256_addlength_spec :=
 DECLARE _SHA256_addlength
 WITH len : Z, c: val, n: Z
 PRE [ _c OF tptr t_struct_SHA256state_st , _len OF tuint ]
   PROP ( 0 <= n+len*8 < two_p 64) 
   LOCAL (`(eq len) (`Int.unsigned (`force_int (eval_id _len))); 
               `(eq c) (eval_id _c))
   SEP (`(sha256_length n c))
 POST [ tvoid ]
   `(sha256_length (n+len*8) c).

Definition SHA256_Init_spec :=
  DECLARE _SHA256_Init
   WITH c : val 
   PRE [ _c OF tptr t_struct_SHA256state_st ]
         PROP () LOCAL (`(eq c) (eval_id _c))
         SEP(`(data_at_ Tsh t_struct_SHA256state_st c))
  POST [ tvoid ] 
          (`(sha256state_ init_s256abs c)).

Inductive update_abs: list Z -> s256abs -> s256abs -> Prop :=
 Update_abs:
   (forall msg hashed blocks oldfrag newfrag,
        Zlength oldfrag < CBLOCKz ->
        Zlength newfrag < CBLOCKz ->
       (LBLOCKz | Zlength hashed) ->
       (LBLOCKz | Zlength blocks) -> 
       oldfrag++msg = intlist_to_Zlist blocks ++ newfrag ->
   update_abs msg (S256abs hashed oldfrag) 
                  (S256abs (hashed++blocks) newfrag)).

Definition SHA256_Update_spec :=
  DECLARE _SHA256_Update
   WITH a: s256abs, data: list Z, c : val, d: val, sh: share, len : nat
   PRE [ _c OF tptr t_struct_SHA256state_st, _data_ OF tptr tvoid, _len OF tuint ]
         PROP ((len <= length data)%nat; 
                   (s256a_len a + Z.of_nat len * 8 < two_p 64)%Z)
         LOCAL (`(eq c) (eval_id _c); `(eq d) (eval_id _data_); 
                `(eq (Z.of_nat len))
                   (`Int.unsigned (`force_int (eval_id _len))))
         SEP(`K_vector (eval_var _K256 (tarray tuint 64));
             `(sha256state_ a c); `(data_block sh data d))
  POST [ tvoid ] 
         EX a':_, 
          PROP (update_abs (firstn len data) a a') LOCAL ()
          SEP(`K_vector (eval_var _K256 (tarray tuint 64));
              `(sha256state_ a' c); `(data_block sh data d)).

Definition SHA256_Final_spec :=
  DECLARE _SHA256_Final
   WITH a: s256abs, md: val, c : val,  shmd: share, sh: share
   PRE [ _md OF tptr tuchar, _c OF tptr t_struct_SHA256state_st ]
         PROP (writable_share shmd) 
         LOCAL (`(eq md) (eval_id _md); `(eq c) (eval_id _c))
         SEP(`K_vector (eval_var _K256 (tarray tuint 64));
             `(sha256state_ a c);
             `(memory_block shmd (Int.repr 32) md))
  POST [ tvoid ] 
         PROP () LOCAL ()
         SEP(`K_vector (eval_var _K256 (tarray tuint 64));
             `(data_at_ Tsh t_struct_SHA256state_st c);
             `(data_block shmd (sha_finish a) md)).
\end{lstlisting}



\end{document}





