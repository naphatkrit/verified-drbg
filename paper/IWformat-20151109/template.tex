\documentclass[pageno]{jpaper}

\newcommand{\IWreport}{2015}

\usepackage[normalem]{ulem}

\begin{document}

\title{
A Latex Template for Independent Work Reports\\
Version 20151109v1}

\author{Firstname Surname\\Adviser: Firstname Surname}

\date{}
\maketitle

\thispagestyle{empty}
\doublespacing
\begin{abstract}
This document is intended to serve as a sample you can use for independent work reports.  We provide some guidelines on content and formatting.  They are not required, but they might be helpful.
\end{abstract}

\section{Introduction}

This document is a sample you can use for formatting your independent work
report.  An example template with all bib and supporting files can be downloaded
from the IW website at {\em http://iw.cs.princeton.edu}.

\section{Preparation Instructions}

\subsection{Paper Formatting}

There are no minimum or maximum length limits on IW reports.
We are including this template because we think it will be helpful
for citing things properly and for including figures into formatted
text.  If you are using \LaTeX~\cite{lamport94}
to typeset your paper, then we strongly suggest
that you start from the template available at
http://iw.cs.princeton.edu -- this
document was prepared with that template.
If you are using a different software package to typeset your paper,
then you can still use this document as a reasonable sample of
how your report might look.  Table~\ref{table:formatting} is a suggestion
of some formatting guidelines, as well as being an example of how to
include a table in a Latex document.

\begin{table}[h!]
  \centering
  \begin{tabular}{|l|l|}
    \hline
    \textbf{Field} & \textbf{Value}\\
    \hline
    \hline
    Paper size & US Letter 8.5in $\times$ 11in\\
    \hline
    Top margin & 1in\\
    \hline
    Bottom margin & 1in\\
    \hline
    Left margin & 1in\\
    \hline
    Right margin & 1in\\
    \hline
    Body font & 12pt\\
    \hline
    Abstract font & 12pt, italicized\\
    \hline
    Section heading font & 14pt, bold\\
    \hline
    Subsection heading font & 12pt, bold\\
    \hline
  \end{tabular}
  \caption{Formatting guidelines. }
  \label{table:formatting}
\end{table}

\textbf{Please ensure that you include page numbers with your
submission}. This makes it easier for readers to refer to
different parts of your paper when they provide comments.

We highly recommend you use bibtex for managing your references and citations.  You can add bib entries to a references.bib file throughout the semester (e.g. as you read papers) and then they will be ready for you to cite when you start writing the report.  If you use bibtex, please note that the references.bib file provided in the template example includes some format-specific incantations at the top of the file.  If you substitute your own bib file, you will probably want to include these
incantations at the top of it.

\subsection{Citations}

There are various reasons to cite prior work and include it as references in your bibliography.  For example, If you are improving upon
prior work, you should include
a full citation for the work in the bibliography \cite{nicepaper,nicepaper2}.
You can also cite information that is used as background or explanation.  In addition to citing scholarly papers or books, you can also create bibtex entries for webpages or other sources.  Many online databases allow you to download a premade bibtex entry for each paper you access.  You can simply copy-paste these into your references.bib file.

\noindent\textbf{Figures and Tables.} Ensure that the figures and
tables are legible.  Please also ensure that you refer to your
figures in the main text. Make sure that your figures will be legible
in the expected forms that the report will be read.  If you expect someone
to print it out in gray-scale, then make sure the figures are legible
when printed that way.

\noindent\textbf{Main Body.} Avoid bad page or column breaks in
your main text, i.e., last line of a paragraph at the top of a
column or first line of a paragraph at the end of a column. If you
begin a new section or sub-section near the end of a column,
ensure that you have at least 2 lines of body text on the same
column.

\subsection{Ethics}

Your independent work report should abide by the basic standards of scholarly ethics and by the Princeton Honor Code. If you have any doubts about how to cite
other work, how to quote or include text or images from other works, or other issues, please discuss them with your project adviser or with the IW coordinators.

\section{Outline}
The following is a possible outline for your paper.
\subsection{Introduction}
\begin{itemize}
\item Problem statement
\item Motivation and goal...The goal of this project is...
\item Roadmap: The remainder of this paper is organized as follows....
\end{itemize}

\subsection{Related Work}
\begin{itemize}
\item Survey of prior work with similar goals
\item Comparison to your project
\end{itemize}

\subsection{Approach}
\begin{itemize}
\item Key novel idea
\end{itemize}

\subsection{Implementation}
\begin{itemize}
\item Things you implemented.  How you did it?
\end{itemize}

\subsection{Evaluation}
\begin{itemize}
\item Experiment design...
\item Data...
\item Metrics...
\item Comparisons...
\item Qualitative results...
\item Quantitative results...
\item Further results needed...
\end{itemize}

\subsection{Summary}
\begin{itemize}
\item Conclusions...
\item Limitations...
\item Future work...
\end{itemize}


\bstctlcite{bstctl:etal, bstctl:nodash, bstctl:simpurl}
\bibliographystyle{IEEEtranS}
\bibliography{references}

\end{document}
